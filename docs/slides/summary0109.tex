\documentclass[a4paper, 12pt]{beamer}

%color theme 
\beamertemplatenavigationsymbolsempty
%\usetheme{metropolis}


\defbeamertemplate*{footline}{shadow theme}
{%
  \leavevmode%
  \hbox{\begin{beamercolorbox}[wd=.5\paperwidth,ht=2.5ex,dp=1.125ex,leftskip=.3cm plus1fil,rightskip=.3cm]{author in head/foot}%
  \usebeamerfont{author in head/foot}\insertframenumber\hfill\insertshortauthor
  \end{beamercolorbox}%
  \begin{beamercolorbox}[wd=.5\paperwidth,ht=2.5ex,dp=1.125ex,leftskip=.3cm,rightskip=.3cm plus1fil]{title in head/foot}%
    \usebeamerfont{title in head/foot}\insertshorttitle%
  \end{beamercolorbox}}%
  \vskip0pt%
}

\usepackage[utf8]{inputenc}
\usepackage{float}
\usepackage{amsmath}
\usepackage{graphicx}
\usepackage{appendixnumberbeamer}
\usepackage{booktabs}
\usepackage{threeparttable}
\usepackage{makecell}
\usepackage{cancel}
\usepackage{algorithm}
\usepackage{algpseudocode}
\usepackage{mathtools}
\DeclarePairedDelimiter{\floor}{\lfloor}{\rfloor}

\title{Status Summary}
\date{01.09.2021}
\author{Lucas Cruz Fernandez}

\begin{document}

\begin{frame}
    \maketitle
\end{frame}

\begin{frame}{Overview}
  \tableofcontents
\end{frame}

\section{Approach}
\begin{frame}{Econometric Model}
  \begin{itemize}
    \item want to estimate the Partially Linear Model
    \begin{align} 
    Y_{it}&=\theta(X_{it})D_{it}+g(X_{it}, W_{it})+\epsilon_{it} \\
    D_{it}&=h(X_{it}, W_{it})+u_{it}
    \end{align}
    \item $\theta(X_{it})$ is the constant marginal CATE
    \item this allows calculating MPC for each household in each period based on their $X_{it}$
  \end{itemize}
\end{frame}

\begin{frame}{Estimator}
  \begin{itemize}
    \item original DML only reasonable for cross-section
    \item JPS and MS also only use cross-section estimators
    \begin{itemize}
      \item[$\rightarrow$] implicitly assume strict exogeneity, which is unreasonable
    \end{itemize}
    \item use Panel Double Machine Learning Estimator (DML) by Chernozhukov et al. (2021)
    \item since treatment dimension fixed ($d_T=1$) only difference to DML is first stage cross-fitting algorithm 
    \begin{itemize}
      \item[$\rightarrow$] see Section 3.1 of Chernozhukov et al. (2021)
    \end{itemize}
  \end{itemize}
\end{frame}

\begin{frame}{Estimator}
  \begin{itemize}
    \item rewrite the PLM to 
    \begin{align*}
      Y_{it}-E[Y_{it}|X_{it}, W_{it}]=\theta(X)(D_{it}-E[D_{it}|X_{it}, W_{it}]) +\epsilon_{it}
    \end{align*}
    \item use a machine learning estimator to find estimators for
    \begin{align*}
      d_{i0}&=E[Y_{it}|X_{it}, W_{it}] \\
      l_{i0}&=E[D_{it}|X_{it}, W_{it}](=h(X_{it}, W_{it}))
    \end{align*}
    \item get respective residuals and estimate $\theta(X)$ using OLS
  \end{itemize}
\end{frame}

\begin{frame}{Panel DML}
  \begin{algorithm}[H]
    \caption{Panel DML Recipe}
    \begin{algorithmic}[1]
        \State Partition the data into K-folds based on their time index. An observation is added to partition $I_k$ if: 
                $$I_k=\{(i, t) : \floor{T(k-1)/K}+1 \leq t \leq \floor{Tk/K}\}$$ 
        \State For each partition $k$ use a first stage estimator to estimate $\hat{d}_k$ and $\hat{l}_k$ using data of all folds except $k$ (cross-fitting). 
        \State Orthogonalize treatment and outcome of observation $(i, t)$ using the predictions of the corresponding fold to get the residuals. 
        \State Use all residuals to estimate the coefficient $\theta$ using a suitable estimator.  
    \end{algorithmic}
  \end{algorithm}
\end{frame}

\section{Problems, Ideas, Questions}
\begin{frame}{Panel DML}
  \begin{itemize}
    \item Panel DML needs lag structure but only have at most 3 observations of $i$
    \item when only using $i$ that have $T=3$ sample size reduced drastically 
    \item one-period lag should be sufficient as looking at quarterly data, hence two periods for each household in reduced dataset
    \item idea: run two (three) specifications
    \begin{enumerate}
      \item without lags and larger N
      \item with lags and smaller N 
    \end{enumerate}
    \item compare whether effects differ strongly 
    \item advantage: Panel DML actually captures these dynamics compared to approaches by JPS and MS
  \end{itemize}
\end{frame}

\begin{frame}{What are X?}
  \begin{itemize}
    \item choosing $X=Z$ leads to undertermined covariance matrix of the estimator 
    \item hence, what variables should be X? 
    \item data-driven way not feasible as this could change derived inference results 
    \item use economic reasoning but what is reasoning then? Could find something for every variable I guess... 
  \end{itemize}
\end{frame}

\begin{frame}{Single variable effects}
  \begin{itemize}
    \item estimate the constant marginal CATE $E[Y_1 - Y_0|X=x]$ 
    \item $X$ is a vector of variables, hence cannot see single variable effects on treatment effect 
    \item idea 1: simply look at correlations between point estimates and variable 
    \item idea 2: use the Marginal Effect at the Means, i.e. setting all $X_{\setminus j}$ at their cross-sectional mean to get $X_j$'s marginal CATE at the mean 
    \begin{itemize}
      \item[$\rightarrow$] major issue: depends extremely on test/train sample split, no matter what way is chosen to train model
      \item[$\rightarrow$] why? salary and total family income have a giant variance
    \end{itemize}
  \end{itemize}
\end{frame}

\end{document}