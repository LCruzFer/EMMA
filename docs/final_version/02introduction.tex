\section{Introduction} \label{sec:intro}
How do households respond to income shocks? And how do their responses differ given their personal characteristics and economic circumstances? These questions are not only at the center of a wide academic debate but also of major importance for policymakers. The former revolves around verifying or negating the main mechanisms of the \textit{Permanent Income Hypothesis} (PIH). Meanwhile, policymakers are interested in improving the efficiency of government transfers. These two sides have sparked many investigations using a wide array of approaches to quantify households' responses to income shocks - the \textit{Marginal Propensity to Consume} (MPC). \\
At the center of macroeconomics since Keynes put it forward as a main driver in his General Economic Theory, the MPC quantifies the amount households will spend on consumption of each dollar they receive from an income shock. Research has long focused on testing whether the MPC out of income shocks is zero and, thus, in line with the PIH, but more recently the focus has shifted. Most studies still support the notion of an average zero MPC, but evidence suggests that for some groups, the response is significantly different. \\
Empirical research related to this heterogeneity in the MPC uses various settings to identify income shocks. Following \cite{parkeretal_2013} and \cite{misrasurico_2014}, we exploit the 2008 tax rebate in the USA to estimate households' MPC through data collected by the \textit{Consumer Expenditure Survey} (CEX). Similar to these two prior studies, we are able to use the rich information on consumption the CEX provides to identify heterogeneities in the overall MPC as well as to analyze the categories of consumption goods most affected by the rebate. In contrast to previous work, our econometric approach sets us apart as it is better suited to detect heterogeneities compared to any contribution we are aware of. \\ 
It is important to highlight that the Economic Stimulus Act in 2008 was signed into law by President Bush in February 2008. The tax rebate payments, which were part of this policy, started in April of the same year and are therefore an anticipated income shock. Also, the tax rebate was disbursed to USA taxpayers during a time of national and global economic downturn. Many households receiving the stimulus might have been in economic turmoil when receiving the payment and actually spend it to cover regular expenses that they otherwise would not have been able to cover (e.g., rent, utilities, or other necessities of daily life). \\
As \cite{kaplanviolante_2014} note, the tax stimulus is anticipated and is subject to these special circumstances. Therefore, one might also speak of our estimated coefficients as a propensity to consume the rebate or \textit{rebate coefficient}, which is not necessarily equivalent to households' overall MPC. While a government stimulus program might not be perfectly appropriate to verify theoretical models concerned with the MPC, providing evidence on their effect on individuals is of major importance for future policymaking. When economic relief is urgent, non-targeted stimuli can present a viable option; targeted transfers can play a major role in many policy settings. Thus, understanding which households use government transfers for consumption and what these households spend them on is an important part of efficient policymaking. Additionally, quantifying the effectiveness of untargeted transfers is necessary to assess whether they are helping to boost the economy. Aggregate estimates of the MPC suggest that this is not the case, but taking a closer look and adjusting for household characteristics reveals heterogeneities and effectiveness of these transfers. Moreover, \cite{parkeretal_2013} emphasize that some rebates were reported to be received outside of the disbursal window, which suggests that the income shocks might not have been anticipated and only noticed after their arrival. \\ 
We use the \textit{Double Machine Learning Framework} (DML) developed by \cite{DML2017} to estimate individual-level point estimates of the MPC as well as standard errors for each household. This enables us to run hypothesis tests for each household whether their MPC is significantly different from zero. DML allows us to estimate the \textit{Conditional Average Treatment Effect} (CATE) of the tax rebate on changes in consumption. Thanks to the semi-parametric nature of the DML framework, we can use reliable machine learning estimators to control for any confounding factors without having to define their relationship with the outcome. Moreover, we present an estimator that retrieves the CATE without assuming a specified relationship between variables we condition on and the CATE itself. \\
Our results underline the heterogeneity of the MPC out of the tax stimulus documented in Parker et al. and Misra and Surico. We find that a large mass of households shows no significant reaction upon receiving the stimulus payment, whereas a smaller fraction of households shows strong and significant reactions above an estimated MPC of 0.5. Compared to previous contributions, we find a smaller share of households showing a significant response while this share shows stronger responses than reported in prior work. Analysing the drivers of the predicted MPC heterogeneities, we show that liquidity is the main one and that low liquidity households are the ones reacting the most. \\
Contrary to prior work, our estimated CATE does not rely on specifying subsets of the data across which we assume heterogeneity to exist. We employ modern methods to quantify the effects of single variables on the estimated MPCs to understand the role of these characteristics. Our non-linear estimators suggest that the heterogeneities presented in other work do not capture the full picture. Next to our contribution to the MPC literature by providing an empirically more robust analysis, we also see our contribution in introducing modern and flexible estimation approaches to the macroeconomic literature. Frameworks such as the DML offer a gateway to new methods and identifications in the macroeconomic literature. We stress the importance of further research into the theoretical and applied nature of these procedures and their usage in more settings. \\
The rest of the paper is structured as follows: Section \ref{sec:lit} summarizes the theoretical and empirical literature on MPC heterogeneity. Section \ref{sec:data} discusses the data source and challenges connected with it. The empirical methodology we use is described in Section \ref{sec:methodology}, while Section \ref{sec:estim_res} presents the identification and estimation results of the MPC. We further investigate sources of heterogeneity in responses in Section \ref{sec:roots_of_heterogeneity}. Section \ref{sec:conc} concludes.