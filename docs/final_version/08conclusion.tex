\section{Conclusion} \label{sec:conc}
We find substantial heterogeneity in the \textit{Marginal Propensity to Consume} out of the 2008 tax rebate among US households. Our results support the notion that a households' financial status, especially its liquidity, play a major role in their decision-making. Our contribution is twofold: for one, we estimate the conditional MPC out of the tax stimulus in a rigorous manner that allows us to detect heterogeneities. Second, we use an estimator that exploits the power of machine learning methods for causal inference and contribute to the wider understanding and promotion of such methods among applied researchers. Machine Learning predictors are powerful tools when it comes to handling large data and detecting complex relationships. \\ 
Our work strongly suggests that prior contributions, while going in the right direction, have missed heterogeneous patterns in the data due to their restricted estimation approaches or lack of better data. We most likely suffer from a similar fate when it comes to the latter. A more detailed and frequent availability of households' characteristics and financial standings (bad word) would allow us to more cleanly estimate the heterogeneous responses. We leave this as a goal for future work. \\
From a policy perspective, it might not be of interest on which exact sub-category of consumption people spend their rebate on - at least when being interested in providing a stimulus to the economy overall. Still, our analysis reveals useful information for making stimuli more targeted to be more effective or to get a general sense of what people spend additional income on, given their characteristics. Although natural experiments such as the 2008 tax stimulus and connected analysis have mostly little external validity outside of their context, the heterogeneity analysis provides a hint on spending patterns and reactions of households given their personal characteristics and financial circumstances. This can act as a starting point when designing more targeted transfer programs. 