\section{Main channels and literature} \label{sec:lit}
There is a vast literature on the marginal propensity to consume and the many factors that potentially play a role in its heterogeneity across households.\footnote{A benchmark literature review of the literature up to their publication can be found in Jappelli and Pistaferri (2010).} Over time, one channel has been identified as one of the key drivers, which is the access to liquid assets. In this section, we want to briefly outline how liquidity drives the MPC of households and discuss the more recent empirical literature related to our study. \\ 
The role of liquidity is linked to the nature of the income shock and borrowing constraints. Within a calibrated life-cycle model, \cite{kalpanviolante_2010} show that households have no access to credit markets, and can therefore not borrow, react substantially more to transitory income shocks than households without such constraints. In general, if a positive income shock is anticipated, households that are already close to or at their borrowing constraint cannot borrow new funds to smooth consumption in anticipation of higher future income. Thus, once the shock is realized, we will observe an increase in consumption. On the other hand, saving is always possible for any household, and hence we will not see a reaction once the shock is realized in case of a negative anticipated shock. Thus, more liquid households react less to a positive anticipated shock in comparison with liquidity constrained ones. In contrast, in case of an unanticipated shock, we expect the opposite. Think of an agent that is temporarily out of work and has no liquid wealth at their disposal. In case of a negative shock, the agent is forced to adjust their consumption behavior downward. Meanwhile, a positive shock will always be saved and stretched over future periods, no matter the level of households' liquidity. \cite{bunnetal_2018} document this asymmetry in reactions to positive and negative shocks using British data.\\
The applied empirical literature investigating the Marginal Propensity to Consume can be categorized into two strands. The first uses data on households' expenditures and observed income shocks; the second relies on surveys that ask respondents directly for their MPC out of hypothetical or experienced income shocks. \cite{parkersouleles_2019} coin these approaches the \textit{revealed preferences} and the \textit{reported preferences} approach, respectively. Our contribution firmly sits within the revealed preferences part of the literature. \\
One common approach to identify income shocks is to look at lottery winners. The odds of winning the lottery are so low that a win can be interpreted as an unanticipated income shock. Studies mostly use state lotteries which have a wide range of small amounts that can be won.\footnote{Using lottery winners who win hundreds of thousands or even millions of dollars would be fruitless since the size of the shock is unreasonably large and probably changes the complete underlying choice-set of households. Additionally, sample sizes would be very small.} For example, \cite{fagerengetal_2018} use Norwegian administrative panel data and find that households winning the lottery spend almost half of their win within one year and 90\% after five years. Moreover, the authors report that liquidity and age are the only variables correlated with the MPC after controlling for confounders. However, correlations are a weak measure to assess drivers of the MPC as we cannot assess which of the variables is the driving force. In a similar vein, \cite{golosovetal_2021} construct a dataset of lottery winners for state lotteries in the USA. They report an average annual MPC of 60 cents out of each dollar won. Supporting the liquidity channel, they find that the highest quartile of the liquidity distribution spends only 49 cents while the lowest quartile spends almost 80 cent of each dollar they win in the lottery. However, these two lottery-based approaches suffer from the drawback that they do not measure consumption directly. Instead, they have to either construct consumption out of households' balance sheet data \citep{fagerengetal_2018} or model consumption as a function of their observed variables \citep{golosovetal_2021}. \\
\cite{fusteretal_2021} provide evidence that households who show strong reactions to unanticipated income shocks show no reaction to news about future gains. Additionally, their approach reveals an intensive and extensive margin of the MPC. Mixing the \textit{revealed} and \textit{reported preference} approaches they show that as the size of the windfalls gains increases, more households report that they would increase their spending. However, among households that reveal a significant reaction to shocks, the reaction is actually declining with the size of the shock. Overall, the extensive margin effect dominates in that more households are spending a significant amount of payments. \\
Recent examples of the reported preferences approach are \cite{bunnetal_2018} and \cite{christelisetal_2019}. The former use data from the \textit{Bank of England} (BoE) to estimate the MPC of British households to income shocks. In the BoE survey, participants are asked about past income shocks they experienced and how they reacted. Their results further support the liquidity channel as the important driver behind MPC heterogeneity. In a theoretical exercise, they show that a model with occasionally binding borrowing constraints can replicate their results. \cite{christelisetal_2019} use Dutch data where they find an average MPC of 15\% to 25\%. Their results reveal strong heterogeneity in responses. Of the households in their sample, 40\% react the same to positive and negative shocks, while another 40\% respond asymmetrically. The latter suggests a strong role of liquidity in decision-making making how to use income shocks. The remaining 20\% of households reveal asymmetric behavior in the opposite direction, which the authors connect to other behavioral models and the lack of financial sophistication. This is in line with \cite{parker_2017}, who also finds a strong relationship between MPC and sophistication as well as other personal traits. He exploits the \textit{Nielsen Consumer Panel} to investigate the 2008 tax stimulus. The \textit{Nielsen Consumer Panel} provides a large set of questions about how households used the rebate received through the tax stimulus and on their personal preferences. Parker estimates that the MPC out of the rebate is 1.5\% of the stimulus received within one week of receipt and 3.5\% within the first four weeks. Based on his findings, Parker suggests that the relationship between liquidity and MPC is not situational but rather dependent on the persistence of low liquidity. I.e., households that only have low liquidity situationally do not respond differently than other households. \\
The already mentioned \cite{parkersouleles_2019} evaluate the two different approaches of the applied literature. They summarize that households that self-report a larger propensity to spend income shocks indeed have larger estimates using the revealed preferences approach. Interestingly they show that self-reported MPCs do not vary conditional on liquidity, the channel which is supported the most by existing studies. \\
Lastly, we want to discuss the two studies most closely related to this paper: \cite{parkeretal_2013} and \cite{misrasurico_2014} (henceforth MS). The former collaborated with the \textit{Bureau of Labor Statistics} (BLS) to add questions about the 2008 tax rebate to the CEX in 2008 and 2009. They estimate the average response of households using OLS and 2SLS, where they instrument the rebate amount with a dummy signaling rebate receipt. Their motivation behind the latter will be subject to discussion when we present our identification strategy in section \ref{subsec:identification}. Estimating various specifications, Parker et al. find MPCs out of the tax rebate that range between 12\% and 30\% when considering changes in non-durable consumption. Taking into account all expenditure categories reported in the CEX, they even find responses between 50\% to 90\%. These higher estimates can be traced back to a small set of households making large purchases in the new vehicle category - a phenomenon that is also documented in MS and our own results. \\
To investigate heterogeneity, Parker et al. follow an approach common to the literature. They define thresholds along the distribution of liquidity and create dummy variables that signal to which group observation $i$ belongs. In their case, they set the cutoff points along the liquidity distribution to cut it into terciles such that each group contains the same amount of households receiving the rebate in a given quarter. Parker et al. report that indeed lower liquidity households show a stronger response to receiving the tax rebate across all their specifications. However, as Misra and Surico point out as well, they only rely on the economic significance but cannot reject the null that the point estimates across groups are significantly different from each other. \\
This approach is similar to another one often used in the applied literature, which is splitting the sample into subsamples and estimating the MPC within each of these. Both these approaches have a severe drawback when it comes to detecting heterogenous patterns. For example, Parker et al.'s approach will only reveal whether whether heterogeneity exists between the three terciles, but any patterns within these terciles are ruled out by construction. In that manner, our estimation approach is superior as we are able to find heterogeneous patterns without pre-defining any sub-groups. \\
Meanwhile, \cite{misrasurico_2014} use quantile regression on the same data to estimate the conditional distribution of the marginal propensity to consume. They find a distribution across all consumption categories that supports the notion by \cite{kaplanviolante_2014} that a large amount of households shows no significant response, while a substantial share has an MPC of around 0.5, and some households react even stronger. In fact, the lower and the upper end of the consumption change distribution are reacting significantly differently from zero, where the reaction is increasing along with the distribution. These findings are present across all consumption categories they investigate. They then check how specific variables - such as liquidity or home-ownership rates - are distributed across the distribution of consumption change. They show that in areas where households show significant reactions to tax rebate receipt - at the lower and upper end of the distribution - the median income is higher than in the center part of the distribution. They conclude that high-income households have either strong positive or zero reactions, while low-income households show a consistently positive reaction of 10\% to 40\%. This explains previously contradicting findings by \cite{sahmetal_2010} and others, who provide evidence on high-income households having the largest MPC, and \cite{jps_2006} and \cite{parkeretal_2013}, who find that low income is associated with higher MPCs. However, these findings are only based on correlations that do not necessarily imply a causal relationship between these factors. \\ 
The results presented in MS must be taken with a grain of salt, too. They ignore a fundamental assumption of quantile regression, which is necessary to interpret their estimates as the MPC and draw the conclusions they present. The assumption in question is the \textit{rank-invariance}, or \textit{rank-preservation}, assumption. We will not discuss the inner workings here, but let us briefly lay out how MS interpretation relies on this restrictive assumption. The coefficient in a quantile regression when estimating the $\tau^{th}$ quantile of the outcome signals how much a one-unit change affects this quantile of the outcome's distribution - in our case, consumption change. However, individuals in this quantile before and after treatment need not be the same. Actually, we would exactly expect the opposite if some individuals react strongly to receiving the rebate and others do not. If, for example, individuals previously did not change their consumption much but now react strongly, they are part of a different quantile than before, and the coefficient only reveals how much the $\tau^{th}$ quantile changes - e.g., because households reacting strongly move out of it. Therefore, their coefficients do not reveal the MPC as we do not look at individuals but only at the distribution. Assuming rank-invariance implies that the rank in the distribution before and after treatment stays the same, and we thus compare the same individuals within quantiles. As we have laid out already, this is not reasonable to assume in our setting and is actually quite counterintuitive. Keeping this issue in mind, Misra and Surico's results can still be helpful when we are interested to understand the distributional effects of the tax rebate. In that light, their results provide evidence for a wider dispersion of the distribution of consumption change after the tax stimulus program.