\section{Understanding the roots of heterogeneity} \label{sec:roots_of_heterogeneity}
In the previous section, we discussed the conditional average treatment effect of each individual given their specific set of characteristics. Prior contributions have looked at the correlations between the significance of the estimated MPC and households characteristics to get a glimpse into which factors play a role in the MPC. However, this approach does not reliably tell us which variables really drive the response. The correlation might very well be spurious or driven by other factors that are correlated with the characteristic we are looking at. Therefore, it is more fruitful to look at measures that can help us identify what role a variable plays in our predicted MPCs. In the case of specifications using the linear DML estimator, we know that this relationship is linear by construction since the CATE is defined as a linear combination of the single effects of interactions between treatment and the respective variable (see equation (\ref{eq:CATE})). However, the causal forest-based approach will help us reveal whether there are any non-linear patterns underlying the effect of characteristics on the MPC without assuming any functional form of these patterns. \\
For this, we turn to the Machine Learning literature, which has developed a number of tools to analyze the relationship between prediction and feature. Feature is a different term for control variables used in the Machine Learning literature. In our setting, these are the variables we condition on to find the CATE, i.e., variables contained in $X_{it}$. Since variables in $W_{it}$ are assumed to not impact the CATE, they are not contained in the second stage and therefore play no role in predicting individuals' MPC.

\subsection{Marginal and Partial Dependence Plots}
\textbf{this section is quite long for something I do not show, right?} \\
Two popular approaches are marginal plots (M-Plots) and Partial Dependence Plots (PDPs; Friedman, 2001). Both use the same general idea to quantify the impact of some feature $x_S$ on our predictions: we replace the value of $x_S$ of each observation with some value $v_1$. Then we fit our trained prediction model to this "counterfactual" dataset and take the average over all these predictions. For example, we predict for each individual what their MPC looks like if they had a certain age and average the predicted MPC. Then we continue with $x_S=v_2$ and so on, where the values $v_j$ are chosen from a grid along the distribution of $x_S$. The difference between M-Plots and PDPs is the distribution of all other features $X_C=X\setminus x_S$ we average over. In the case of Marginal Plots, contrary to what the name might suggest, we use the conditional distribution of $X_C$ given $x_S$, $p_{X_C|x_S}$, to obtain the impact of $x_S$ on our prediction. On the other hand, PDPs use the marginal distribution of $X_C$, $p_{X_C}$. 
% \begin{align}
% \hat{f}_M(x_S=v_j)=\int p_{X_C|x_S=v_j} m(x_S=v_j, X_C)dX_C, 
% \end{align}
% where $m$ is our predictor and $\hat{m}_M(v_j)$ is the effect at $x_S=v_j$. 
% \begin{align}
%     \hat{f}_{PDP}(x_S)=\int p_{X_C} f(x_S, X_C)dX_C \label{eq:pdp}
% \end{align}
% where $\hat{m}_{PDP}(v_j)$ is the Partial Dependence of our predictor on $x_S$ at $v_j$. 
Partial Dependence Plots are more common in the Machine Learning literature as M-Plots suffer from a severe weakness when features in $X_C$ are correlated with $x_S$. However, PDPs also fail to reliably uncover the effect of $x_S$ in such a setting. \\
To illustrate the issues arising in M-Plots and PDPs when features are correlated, let us consider a simple example. Let's say we have some predictive model $m$ that only depends on two predictors $x_1$ and $x_2$, which are positively correlated. To now calculate the M-Plot of $x_1$ at $v_1$ we use the conditional distribution $p_{x_2|x_1}$. In practice, we plug in $x_1=v_1$ for each observation that is within a specified neighborhood of $x_1=v_1$ (e.g. observations in the same quantile). Then we predict and average to obtain the M-Plot value at $x_1=v_1$. Repeating this procedure for other values $v_j$ then results in the M-Plot of $x_1$. However, because the two variables are correlated, we do not know which variable drives the observed effect - if $x_1$  is increased, the values of $x_2$  we use for our predictions also increase because we only use $x_2$ of observations that are close to having $x_1=v_j$. This problem is known as 'conflation.' \\
On the other hand, Partial Dependence Plots do not suffer from this problem because they use the marginal distribution of $x_2$. We use all observations of $x_2$ instead of only looking at a neighborhood in which $x_1=v_1$ and, therefore, average out the effect of $x_2$ on our predictions. Since we use the same set of $x_2$ values at each point $v_j$, we know that changes in our predictions must stem from $x_1.$ Still, the PDPs are not a good tool when features are correlated, and this is connected to the machine learning estimators we apply them to. These are nonparametric estimators that are usually very weak in predicting outcomes based on observations they have never seen before. This extrapolation, however, becomes necessary when we create the "counterfactual" dataset by setting $x_1=v_j$. By doing so, we effectively create observations that are extremely unlikely or even impossible to observe in the real world because of the correlation between the features. For example, in our data, age and salary are strongly correlated, which is quite intuitive because once retired, households do not receive a salary anymore. When creating PDPs, we ignore this fact and create households that have a high salary and are very old. The weakness in extrapolation leads the model to create imprecise predictions, which then severly bias the Partial Dependence Plots. (Apley and Zhu, 2020)  \\
Therefore, while PDPs do not suffer from theoretical drawbacks like M-Plots, in practice, they are unable to uncover the effects of $x_1$ on our predictions in a stable manner because of the underlying predictive estimator. If the true model is indeed linear and we use a linear prediction method with the correct specifications of any interaction terms etc., then this extrapolation issue is unlikely to occur. Moreover, by construction, a linear predictor will result in linear Partial Dependence Plots. 

\subsection{Accumulated Local Effects}
To circumvent issues arising in M-Plots and PDPs from correlated features, Apley and Zhu (2019) propose Accumulated Local Effects (ALE). The extrapolation issue PDPs suffer from is bypassed by using the conditional distribution $p_{X_C|x_S}$ as we do in M-Plots. The 'conflation' effect that results from this is tackled by not using average predictions at $x_S=v_j$ but rather the average marginal change in predictions at this point. In other words, we use the partial derivative of our predictor in question $m$ with respect to $x_S$ at the point $v_j$. Although tree-based machine leanring methods such as the causal forest have no concept of a gradient, Apley and Zhu are able to derive proofs for non-differentiable predictors $m$. Further does the gradient play no role when actually estimating the ALE as it is approximated by a step-function. The ALE is defined as
\begin{align}
\hat{m}_{S, ALE} (x_S)=\int_{z_{0, S}}^{x_S} E_{X_C|X_S=x_S}[\hat{m}^S(X_S, X_C)|X_S=z_S]dz_S - constant. \label{eq:ale}
\end{align}
Looking at this equation step-by-step reveals how the ALE recovers the effect of $x_S$ on our predictions even when features are correlated. As already mentioned, the ALE avoids 'conflation' by using the partial derivative of $m$, where we have $m^S=\frac{\partial m}{\partial x_S}\rvert_{x_S=v_j}$ as the partial derivative of $m$ evaluated at the point we want to find the ALE for. Since we only look at an infinitesimally small change, this change in $x_S$ will not affect the features that are correlated with it in $X_C$ unless the correlation is extremely high. In our analysis, we would want to avoid this case anyways to avoid problems in the estimation itself (e.g., multicollinearity). Once the changes in prediction are obtained for each observation, we average them over the conditional distribution, i.e., only using observations that are within a neighborhood of $x_S=v_j$ and actually exist. Now we have the average local effect of $x_S$ on our prediction. To better visualize the global role of this feature, Apley and Zhu argue that this can be achieved by accumulating all local effects up to $x_S=v_j$. Thus, we simply integrate overall local effects up to $x_S=v_j$, where $z_{0, S}$ is the lower bound of the distribution of $x_S$.\footnote{For more on this, see Section 5.2 of Apley and Zhu (2019) where they demonstrate how accumulation helps to improve the interpretability of ALE plots.} \\
To estimate the ALE we use the following estimator, which illustrates the procedure in more intuitive terms: 
\begin{align}
\hat{\tilde{m}}_{j, ALE}(x_S=v_j)=\sum_{k=1}^{k_j(x_S=v_j)} \frac{1}{n_j(k)}\sum_{i:x_{i,j}\in N_j(k)}[m(z_{k,j}, x_{i,\setminus j})-m(z_{k-1,j}, x_{i,\setminus j})] \label{eq:ale_estim}
\end{align}
First, we bin our data into $n_b$ bins based on quantiles of the distribution of $x_S$. To mimic the marginal change represented by the partial derivative $m^S$ in \ref{eq:ale} we make two predictions for each individual. For an observation $i$ that falls in bin $k$, we predict its outcome with $x_S=z_k$ and $x_S=z_{k-1}$, where $z_k$ represents the upper bound value of bin $k$. We then average over all individuals that fall within this bin $k$ denoted in equation (\ref{eq:ale_estim}) by the neighborhood $N_j(k)$. Finally, we accumulate all predicted preferences over all bins up to the bin at which point x falls, denoted by k(x). Only looking at individuals within the neighborhood $N(k)$ accounts for the conditional distribution used in (\ref{eq:ale}). \\
Additionally, Apley and Zhu propose to center the effect around the average of all ALEs such that the mean effect is zero. Then, the ALE has to be interpreted relative to the average prediction, and it shows whether for a given $x_S=v$ the effect of $x_S$ is above or below the average prediction. I.e., whether $x_S$ affects our predictions at $x_S=v$ more than it does on average. In practice, the $constant$ in (\ref{eq:ale}) is replaced by 
\begin{align*}
    \frac{1}{n}\sum_{i=1}^n \hat{m}_{j, ALE}(x_{i, j}).
\end{align*}
Note that we yet cannot say something meaningful about the statistical significance of these results. Most fields are only interested in the predictive power of machine learning methods and in understanding how these predictions are achieved, but there is no notion of statistical significance in these settings. Therefore, a specific approach to quantify the uncertainty of these measures has not yet been developed. While a deeper look is out of scope of this paper, we use bootstrapping to briefly dive into the topic of statistical significance. We simulate the ALEs for $n_{bootstrap}$ samples. These create an empirical distribution on which basis we can calculate a pseudo confidence interval using the reverse percentile approach (Davison and Hinkley, 1997, p. 194 eq 5.6). We report these as the red lines surrounding the ALE in Figures X.X. We see that these bands are very wide in certain parts - especially in areas where there is a small number of observations. We strongly encourage a deeper investigation of the statistical properties of ALEs and a potential way of quantifying their uncertainty in a more rigorous manner. While the ALEs show us the relationship between a specific variable and our predictions, having a sound theoretical foundation would be desirable to better assess the statistical significance of these relationships. 

\subsection{Results} 
To investigate what role single features play in the MPC we mainly focus on the response with respect to changes in non-durable consumption. We will briefly draw comparisons to other consumption categories where we gain important insights from doing so. Moreover, we only discuss the features relevant within the context of the literature. As we will see these are proving to be drivers of the MPC as expected. The ALE figures for all specification-estimator pairs of each of the main consumption categories are presented in Appendix A.A. As pointed out in Section 6.1, by the construction of the estimator and the ALE, it will always depict a linear relationship when looking at the linear DML setting. However, we can still infer in what direction the relationship is going - e.g., whether predictions are above or below average for young people. More importantly, it is useful as a benchmark to compare the ALE of our estimates using the causal forest as the second stage estimator. \\
Comparing the ALE of the causal forest estimator to the linear model helps us understand where non-linearities play an important role in understanding MPC heterogeneity. Still, we have established in Section 5.3 that the causal forest estimator reveals more significant MPCs in specification 3, where we control for liquidity, which is likely to occur because of nonlinearities not picked up in the linear CATE model. This notion becomes evident right away when we compare the ALE of the linear CATE and the non-linear CATE we retrieve with the causal forest estimator. The clear positive relationship between age and MPC we find in the linear estimator immediately breaks down in the causal forest based model. Instead, we see a rather unstable relationship with wider pseudo confidence bands. Younger and middle-aged households (45 to 55) seem to be experiencing lower MPCs than older households. However, only for very narrow settings do our bootstrapped CIs completely lie below zero. This signals a very unstable relationship between age and the MPC overall, which is a strong contrast to the linear DML results. These dynamics are observed throughout all consumption categories. \\
Interestingly, the role of age in the linear model is reversed once we introduce liquid assets, salary and income in Specification 3 (Figure X.X Panel B). The deviations decrease substantiallyand the ALE shows that older households seem to have a lower MPC. Given the correlation between age and each of these variables this underlines their importance since age seems to act as a proxy for them. This bias vanishes by including the \textit{financial status} variables. This dynamic is also upheld in the causal forest estimator, where we see that the effect of age to be more stable and more closely fluctuating around zero. It supports the notion that younger households show a stronger response to the income shock. \\
The main channel identified in the literature is liquidity. Our discussion of the underlying theory of binding borrowing constraints and lacking access to liquid assets provides the intuition for the following analysis. Note that the pattern described here is also present across all consumption categories although the ALE magnitude is varying. The picture painted by the liquidity channel is only in part supported by our results. First, the linear estimator shows a negative relationship between MPC and liquidity only for changes in total consumption. When we turn to non-durables and strictly non-durables, the ALE actually suggests that households with higher liquidity show a stronger positive reaction to the income shock. While these effects are very small, this still suggests that the liquidity channel does not necessarily act as expected in creating MPC heterogeneity. One possible explanation is that for high liquidity households the tax rebate represents a relatively small income shock. Therefore, they do not depend on it financially, are less informed and are taken more by surprise. However, this phenomenon is quickly turned around by the causal forest results. It fully supports the liquidity channel as we find that low liqudity leads to a high MPC, while once a certain threshold in liquid asset holdings is exceeded the MPC is decreased. Hence, the liquidity channel in cases of anticipated income shocks such as the tax rebate is underlined by the causal forest results. Low liquidity households are not able to react at the announcement to the shock and thus show a strong reaction once the tex rebate is received. On the other hand, a higher level of liquidity enables households to increase consumption prior to receiving the rebate and smooth consumption level overall. \\
Interestingly, the causal forest reveals that at the very bottom of the liquidity distribution, households show a negative response to rebate receipt. These negative responses might explain why the linear DML shows no role of liquidity at low levels of liquidity as it does not fully capture non-linearities and interactions, leaving the estimator biased. On the other hand, they are contrary to the main idea of the liquidity channel. This result shows that the clear relationship the liquidity channel paints in theory does not hold up in the data. Instead, the negative effect of very low levels of liquidity on the predicted MPC highlights a new role. A possible explanation might again be related to the circumstances under which the stimulus is received. Households with no liquidity might save up the stimulus to build up a buffer during the times of economic downturn as were taking place in 2008. \\
\textbf{here tran-sition missing} The sharp drop after roughly 2,000 USD suggests that households that were capable of increasing their consumption at announcement by tapping into their liquid assets consequently have a lower MPC out of the rebate as they save it to further smooth consumption by saving it. The drift to a positive role of liquidity once households have high liquid asset holdings is very unstable as evidenced by the very wide CIs that also include negative values and zero across all levels of liquidity. Most likely, this is due to the small amount of households in our sample that report such high levels of liquidity leaving the causal forest performing imprecisely. \\
Lastly, we turn to the other two financial variables, salary and income, which are presented in Figure X.X. In both we see highly non-linear dynamics that appear to be quite similar once a threshold of 50,000 USD is passed. We see that a very low salary leads to a lower MPC, which is reasonable to expect considering the economic circumstances in which the tax rebate took place. Low income households struggling during the recession might show a smaller response because they use the tax rebate for precautionary savings (maybe a citation here of a precautionary savings paper). On the other hand, the MPC is higher for households that have higher salaries in range between 50,000 and 75,000 USD before the effect vanishes at higher levels. This is also in line with what we expect as the tax rebate was phased out at reported incomes above 75,000 USD. Thus, high income households for which the tax rebate plays a minor role in their income anyway also received even less. 