\documentclass[12pt]{article}
\usepackage[utf8]{inputenc}
\usepackage{amsmath}
\usepackage{natbib}
\usepackage{graphicx}
\usepackage{threeparttable}
\usepackage{makecell}
\usepackage{setspace}
\usepackage{geometry}
\usepackage{booktabs}
\geometry{a4paper, left = 25mm, right = 25mm, top = 20mm}
\onehalfspacing

\title{Glossary of Mathematical Terms}
\author{Lucas Cruz Fernandez}
\date{\today}

\begin{document}
    \maketitle 
    \noindent \textbf{This glossary contains notes on mathematical terms/definitions I encounter, deem relevant to understand deeper and might want to look up on a more regular basis.}

    \tableofcontents

    \newpage 
    
    \section{Vectors, Matrices \& Co.}
    
    \subsection{Sparsity}
    Here: Sparse vectors or matrices have many zero elements. We can denote the vector in its sparse representation by only considering the indices that have non-zero elements. \\
    See for example Chernozhukov, Goldman, Semenova and Taddy (2021) on DML for panel data 

    \subsection{Next }
\end{document}