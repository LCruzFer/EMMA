\section{Literature} \label{sec:lit}
The literature investigating the size of the MPC and potential heterogeneity can be summarized in three different strands. \\
The first one uses quasi-experimental settings to identify income shocks and the resulting reaction of consumption. Settings considered are US tax stimulus progrmas during the times of economic crisis in 2001 and 2008 or lottery wins by individuals. Johnson, Parker and Souleles (2006) estimate the size of the MPC out of the 2001 tax stimulus and in a more recent contribution also take a look at the 2008 tax rebate program (Johnson, Parker and Souleles, 2013). The latter is closely related to our procedure and is hence discussed in more detail further below. They find XXX. Meanwhile, Fagereng et al. (2020) estimate the heterogeneous MPC out of lottery winners in Norway. Golosov et al. (2021) do the same using bla data. 
The second strand of literature uses self-reported MPC from household surveys. 
%what are drawbacks of self-reporting 
However, studies based on self-reported data are prone for measurement error - specifically the so-called self-report bias which leads respondents to misreport their data. In the case of the Marginal Propensity to Consume we expect this to be even larger than in the survey data exploited in quasi-experimental settings since respondents do not only have to document their raw spending behaviour (e.g. indicating how much money was spent in total) but assess their MPC on their own. Such calculations are likely to increase the risk of measurement error, especially the more abstract the concept becomes. \\
There is also a more theoretical side to the discussion focusing on calibrating heterogenous agent new keynesian models (HANK) to uncover general equilibrium effects of single agents' MPCs on the aggregate MPC out of income shocks.

%Concluding sentence before JPS and MS discussion 
However, what becomes evident in all strands of the existing literature is the important role liquidity and the size of the shock plays in households' response. 

%make this the end of the literature review elaborating more precisely on their estimation approaches and drawbacks
Finally, there are two contributions that are by default most closely related to our setting since we make use of the same data. Namely, these are Johnson et al. (2013) and Misra and Surico (2014). 
%some draft on JPS and MS problems
They estimate a simple fixed-effects regression in which they interact their income schock variable with pre-defined dummies. Those dummies are based on continuous variables and created by choosing discrete cut-off points. However, this prohibits the detection of heterogeneous patterns that are not captured by the variables considered or are not inside the defined thresholds. Using the Parker et al. data, Misra and Surico (2014) replicate their approach but use quantile regression to analyse the heterogeneity in the MPC distribution. While quantile regression can be of service to detect heterogeneity in coefficients, it does not allow for the correct interpretation. The treatment effects they uncover are the effect of the income shock on the difference in consumption before and after for a respective quantile. However, this quantile does not need to include the same individuals. Hence, the quantile regression only uncovers shifts in the overall distribution but is silent on how specific individuals changed their consumption pattern - and hence the actual MPC. \\
Lastly, as Kaplan and Violante \textbf{(or who exaclty was it?)} point out, empirical analysis that use stimulus payments as a temporary income shock to identify the MPC might actually estimate another coefficient, which they coin the coefficient of rebate. They argue that the conditions of a stimulus payment as well as the overall economic conditions that lead to such a payment are too specific (\textbf{rephrase}) to 

%discussion on strict exogeneity in JPS/MS 
In the JPS and MS specifications, they also consider simple linear estimators (OLS and Quantile Regression) that imply the assumption of strict exogeneity. Since we are looking at quarterly data and JPS/MS only consider age and change in the size of family as confounders, one could argue that there is little to no variation in these variables between quarters. In that case, 