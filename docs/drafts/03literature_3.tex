\section{Liquid Channels of heterogeneity and the current state of the literature} \label{sec:lit}
Jappelli and Pistaferri (2010) provide a simple life-cycle model to illustrate various channels in the theoretical literature that have been identified as potential drivers of MPC heterogeneity. We will briefly summarize the channels most relevant to our analysis before we turn to the more recent empirical literature on MPC heterogeneity. \\
The first channel is the precautionary savings motive. Some households will show no response to income shocks because they tend to save them as a precautionary measure. This is a mechanism we could expect playing a role in our setting because we investigate a tax stimulus paid out during a strong recession. Uncertainty over future outcomes might increase - or be perceived as higher - by households and thus their savings motives increase. However, at the same time we might argue that especially low income households need the stimulus to pay their day-to-day bills. This ambiguity might be one driver of the heterogeneity we observe. \\
However, these precautionary savings by the households themselves are necessary because they do not have access to insurance markets (Carroll, 2001; Blundell et al., 2008; Kaplan and Violante, 2009). These incomplete insurance markets are another driver of MPC heterogeneity. Given heterogenous risks across the spectrum of households, access to insurance markets varies for each household. Hence, households without access rely on precautionary savings to insure themselves against future income shocks. Generalizing, the literature takes a look at any form of credit market to which households' access might be constrained. This brings us to the most global driver identified in the literature, which is the liquidity channel. \\
It is the most important and most prominent in the MPC literature and has been subject to the most and earliest attention (Carroll 1997; Deaton, 1991; and Zeldes, 1989). Its role is linked to the nature of the income shock and borrowing constraints. For example, Kaplan and Violante (2009) show that in a calibrated life-cycle model households that have no access to credit markets and can therefore not borrow react substantially more to transitory income shocks than do households without such constraints. In general, if a positive income shock is anticipated, households that are already close to or at their borrowing constraint cannot borrow new funds to smooth consumption in anticipation of a higher future income. Thus, once the shock realises, we will observe an increase in consumption. On the other hand, saving is always possible for any household and hence we will not see a reaction once the shock realises in case of a negative anticipated shock. Thus, more liquid households react less to a positive anticipated shock in comparison with liquidity constrained ones. In contrast, in case of an unanticipated shock, we expect the opposite. Think of an agent that is temporarily out of work and has no liquid wealth at their disposal. In case of a negative shock, the agent is forced to adjust their consumption behaviour downward. Meanwhile, a positive shock will always be saved and stretched over future periods, no matter the level of households' liquidity. E.g. \cite{bunn_etal} document this asymmetry depending on the sign of the shock. \\
Another potential driver of heterogeneity is households' age. Attanasio and Weber (2010) document the life-cycle dynamics and summarize the 'retirement-consumption puzzle', however, this is concerned with the amount spent overall, not changes in consumption. Hence, while older households might operate on a lower level of consumption expenditures, changes in amounts are not necessarily subject to this. Moreover, we will see that estimates concerning age capture the effects of other confounders, most prominently liquidity, as they are correlated. \\
Lastly, a source of heterogeneity might be individual level differences in preferences, which are subject to investigations in behavioral studies. Especially households patience, represented by their dicsount factor in theoretical models, is of importance when considering reaction to income shocks (need citation of the papers doing this; look for it again in notes).
The applied empirical literature investigating the Marginal Propensity to Consume can be categorized into two strands. The first uses data on households' expenditures and observed income shocks, the second relies on surveys that asks respondents directly for their MPC out of hyopthetical or experienced income shocks. Parker and Souleles (2019) coin them the 'revealed preferences' and the latter the 'reported preferences' approach, respectively. Our contribution firmly sits within the revealed preferences part of the literature. Similarly, many studies have investigated the MPC using expenditure data. One common approach is to exploit surveys of lottery winners. The odds of winning the lottery are so low that a win can be interpreted as an unanticipated income shock. Studies mostly use state lotteries which have a wide range of small amounts that can be won.\footnote{Using lottery winners who win hundreds of thousands or even millions of dollars would be fruitless since the size of the shock is unreasinably large and probably changes the complete underlying choice-set of households. Additionally, sample sizes would be very small.} For example, Fagereng et al. (2020) use Norwegian administrative panel data to estimate the MPC using winners from a national lottery as the treatment group. They find that households winning the lottery spend almost half of their win within one year and 90\% after 5 years. Moreover, the authors report that liquidity and age are the only variables correlated with the MPC after controlling for confounders. However, correlations are a bad instrument to assess drivers of the MPC as we cannot assess which of the variables is the driving force. E.g. age could be correlated with liquidity and only be correlated with the MPC through this liquidity channel. In similar vein, Golosov et al. (2020) construct a dataset of lottery winners of state-lotteries in the USA. While their main goal is to estimate households' labor market response, they also construct a measure to recover the MPC using their estimates. They find that on average the MPC out of lottery wins is 60ct of each dollar won. Further supporting the liquidity channel, they find that the highest quartile of the liquidity distribution spends only 49ct while the lowest quartile spends almost 80ct of each dollar won in the lottery. However, these two lottery-based approaches suffer from the drawback that they do not measure consumption directly. Instead they have to either construct consumption out of households balance sheet data (Fagereng et al., 2018) or model consumption as a function of their observed variables (Golosov et al., 2019). \\
Fuster et al. (2020) invetigate the MPC out of unanticipated gains. They also find a vast heterogeneity in responses. On the other hand, they show that news about future income gains lead to no reaction on the households side. Most interesting and outstanding from other literature, however, is their approach to identify extensive and intensive margins of the MPC. As the size of windfalls gains increases, households report that they would increase their spending (reported preferences approach) but conditional on responding they find that households' reactions decline in the size of gains. Overall, the extensive margin effect is stronger in that more households are spending a significant amount of payments. They calibrate a precautionary savings model that is capable of replicating all their findings. \\
Recent examples of the reported preferences approach are Bunn et al. (2018) and Christelis et al. (2019). The former use data from the Bank of England to estimate the MPC of British households to income shocks. In the BoE survey participants are asked about past income shocks they experienced and how they reacted. Their results further support the liquidity channel as the important driver behind MPC heterogeneity and show that indeed households have asymmetric reactions to positive and negative income shocks. Their estimated average MPC is around 0.46 to 0.86 for negeative income shocks and 0.07 to 0.17 for positive shocks. In a theoretical exercise they show that a model with occasionally binding borrwoing constraints can replicate their results.\\
Christelis et al. (2019) use Dutch data of reported preferences to also investigate the asymmetry of the MPC and the role of the size of the shock. They find an average MPC of 15\% to 25\%. In line with the above described liquidity channel, households react more to negative shocks. Moreover, their results reveal a strong heterogeneity in responses. 40\% of households in their sample react the same to positive and negative shocks, while another 40\% respond asymetrically. The latter suggests a strong role of liquditiy in decision making how to use income shocks. The last 20\% of households reveal asymmetric behavior in the opposite direction, which the authors connect to other behavioural models and the lack of financial sophistication. This is in line with Parker (2017) who also finds a strong relationship between MPC and sophistication. \\
The already mentioned Parker and Souleles (2019) evaluate the two different approaches of the applied literature. They summarize that households that self-report a larger propensity to spend income shocks indeed have larger estimates using the revealed preferences approach. Interestingly they show that self-reported MPCs do not vary conditional on liquidity, the channel which is supported the most by existing studies. \\
Parker (2017) uses data from the Nielsen Consumer Panel to to also investigate the 2008 tax stimulus. The Nielsen Consumer Panel provides a large set of questions about how households used their rebate and on their personal preferences. He finds that the MPC out of the rebate is 1.5\% of the stimulus received within one week of receipt and 3.5\% within the first 4 weeks. He shows that spending responses are strongly related to personal characteristics. For example, households' response is heavily related to self-reported characteristics such as sophistication, planning and impatience. Moreover, he also finds evidence supporting the liquidity channel. Also, whether households report to be savers or spenders is strongly correlated with their MPC. Based on his findings, Parker suggests that the relationship between liquidity and MPC is not situational but rather must a long period of low liquidity persist before it affects the households response. I.e. households that only have low liquidity situationally do not respond differently than other households. \\
\\
Further, Parker et al mention the following papers that have similar findings as they do: Agarwal, Liu and Souleles, 2007; Broda and Parker, 2008; Johnson, Parker and Souleles, 2006 (have to mention that one); Bertrand and Morse, 2009): all (probably) looking into 2001 stimulus; have to mention at least JPS \\
Lastly, we want to discuss the two studies immediately related to this paper: Parker et al. (2013) and Misra and Surico (2014). The former collaborated with the BLS to add questions about the 2008 tax rebate to the CEX in 2008 and 2009. They estimate the average response of households using OLS and 2SLS, where they instrument the rebate amount with a dummy signaling rebate receipt. Their motivation behind the latter will be subject to discussion when we present our identification strategy in section \ref{subsec:identification}. Estimating various specifications, Parker et al. find MPCs out of the tax rebate that range between 12\% and 30\% when considering changes in non-durable consumption. Taking into account all expenditure categories reported in the CEX they even find responses between 50\% to 90\%. These higher estimates can be traced back to a small set of households making large purchases in the new vehicle category - a phenomenon that is also documented in Misra and Surico and our own results. \\
To investigate heterogeneity, Parker et al. follow one baseline approach in the literature. They define thresholds along the distribution of liquidity and create dummy variables that signal to which group observation $i$ belongs. In their case they set the cutoff points along the liqudity distribution to cut it into terciles and that each group has the same amount of households receiving the rebate in a given quarter. Parker et al. report that indeed lower liquidity households show a stronger response to receiving the tax rebate across all their specifications. However, as Misra and Surico point out as well, they only rely on the economic singificance but cannot reject the null that the point estimates across groups are signficantly different from each other. \\
Additionally, similar to another approach present in the literature - splitting the sample into subsamples and estimating the MPC in each sample on its own - this approach can miss out on patterns of heterogeneity. More precisely, we will only find out whether heterogeneity exists between the three terciles but any patterns within these terciles are ruled out by construction. In that manner, our estimation approach is superior as we are able to find heterogeneous patterns without pre-defining any sub-groups. \\
Meanwhile, Misra and Surico (henceforth MS) use Quantile Regression on the same data to estimate the conditional distribution of the marginal propensity to consume. They find a distribution across all consumption categories that supports the notion by Kaplan and Violante (2014) that a large amount of households shows no singificant response, while a substantial share have a MPC around 0.5 and some households react even stronger. In fact, the lower and the upper end of the consumption change distribution are reacting significantly different from zero, where the reaction is increasing along the distribution. These findings are consistent across all consumption categories they investigate. They then check how specific variables - such as liquidity or home-ownership rates - are distributed across the distribution of consumption change. They show that in areas where households show significant reactions to tax rebate receipt - at the lower and upper end of the distribution - median income is higher than in the centre part of the distribution. They conclude that high income households have either strong positive reactions or zero reactions, while low income households show a consistently positive reaction of 10\% to 40\%. This explains previously contradicting findings by Sahm, Shapiro, and Slemrod (2010) and others, who provide evidence on high income households having the largest MPC, and Johnson, Parker and Souleles (2006) and Parker et al. (2013), who find that low income is associated with higher MPCs. However, these findings are only based on correlations that are not necessarily implying a causal relationship between these factors. \\ 
However, Misra and Suricos results must be taken with a grain of salt as they ignore a fundamental assumption of quantile regression, which is necessary to interpret their estimates as the MPC and draw the conclusions they present. The assumption in question is the \textit{rank-invariance}, or \textit{rank-preservation}, assumption. Discussing the inner workings of quantile regression is out of the scope of this work here but we briefly lay out how MS interpretation is flawed as long as we do not make this restrictive assumption. The coefficient in a quantile regression when estimating the $\tau^{th}$ quantile of the outcome signals how much a one unit change affects this quantile of the outcome's distribution - in our case consumption change. However, individuals in this quantile before and after treatment must not be the same. Actually, we would exactly expect the opposite if some individuals react strongly to receiving the rebate and others do not. If for example individuals previously did not change their consumption much but now react strongly, they are part of a different quantile than before and the coefficient only reveals how much the $\tau^{th}$ quantile changes - e.g. because households reacting strongly move out of it. Therefore, their coefficients do not reveal the MPC as we do not look at individuals but only at the distribution. Assuming rank-invariance implies that the rank in the distribution before and after treatment stays the same and we thus compare the same individuals within the quantiles. As we have laid out already this is not reasonable to assume in our setting and actually quite counterintuitive. Keeping this issue in mind, Misra and Surico's results can still be helpful when we are interested to understand distributional effects of the tax rebate - e.g. whether it moves closer together or spreads further apart.