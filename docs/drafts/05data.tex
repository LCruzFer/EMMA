\newpage
\section{Data} \label{sec:data}
We use data collected by the Consumer Expenditure Survey (CEX) that is administered by the Bureau of Labor Statistics. Additionally to the main questionnaire, the CEX added questions about the stimulus payments to their surveys conducted between June 2008 and March 2009 (**is this correct time in dataset?**). The main advantage of the CEX is that it creates a unique representative sample that contains finely grained information on the type of goods households consume. While its main purpose is to serve as the benchmark to determine the goods basket used to measure inflation in the USA, it enables a detailed analysis on what households spent their rebate on. In the following, we briefly outline the stimulus program and describe the CEX data. 

\subsection{The 2008 Tax Stimulus Program} 

Due to the global financial crisis and the subsequent recession, the United States government passed the Economic Stimulus Act (ESA) in February 2008. With projected costs of more than 150 billion USD it was the largest relief program passed in the USA's history. Next to the stimulus payments, which made up roughly two thirds of the program, the ESA also enacted other steps meant to provide economic relief such as enabling government owned entities (Fannie Mae and Freddie Mac) to buy up more mortgages. However, we focus on the effects of the stimulus payments. 

Each household that filed for taxes in 2007 was eligible to receive a minimum amount of 300 USD (600 USD for couples filing jointly) and a maximum of 600 (1200) USD. The exact amount of each individual was determined by their net tax liability in 2007 up to the maximum amount. Additionally, households received 300 USD per dependent child under the age of 17. Meanwhile, the payment was also connected to households gross income, with the rebate being reduced by 5\% of the amount the income exceeded 75,000 (150,000) USD. (**the last two sentences are very close to wikipedia source**) 

\subsection{Consumer Expenditure Survey} 

The CEX is a representative survey of households in the USA. Once a household is selected to participate, they are interviewed a total of five times. The first interview is a baseline interview in which some general household characteristics such as the financial circumstances are determined and their stock of nondurable goods are documented. The next four interviews are administered every 3 months in which households are asked to document their expenditures over these past three months. After that, the household is rotated out of the CEX and replaced with a new one. Hence, each month of data documented in the CEX contains a different set of households as new ones are added and others do not show up anymore as they are rotated out of the survey.