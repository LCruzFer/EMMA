\section{Estimation and Results} \label{sec:estim_res}
We implement the partially linear model as presented in Section X.X to estimate the effect of receiving a tax rebate of $R_{it}$ on change of consumption $\Delta C_{it}$
\begin{align}
    \Delta C_{it+1}&=\theta(X_{it})R_{it+1}+g(X_{it}, W_{it})+\epsilon_{it} \label{eq:plm_C1}\\
    R_{it}&=h(X_{it}, W_{it})+u_{it} \label{eq:plm_C2}
\end{align}
Which variables are included as confounders $X_{it}$ and $W_{it}$ and which make up $X_{it}$ depends on our specification and is named in the section discussing the respective results. In each specification we include monthly dummies to account for seasonality. They also capture any unobserved effects that might appear when households learn about the rebate, i.e. in line with Parker et al./Misra and Surico our estimation uncovers the effect of actually receiving the rebate and not the global rebate effect.

\subsection{Identifying the Income Shock}
Since we use the same data and event to estimate the MPC our identification is based on the approach by Parker et al. (2013). The main factor is the design of the stimulus rollout, which we can exploit to identify the income shock. The tax stimulus was paid out to households over several weeks as administrative and technological restrictions made it impossible to pay out all rebates at once. Instead the rebate receipt depended on the last two digits of tax filers' social security number. Therefore, we observe rebate receipts at different points in time, which allows us to use all other households that received their rebate in a different quarter as the control group. \\
\textbf{the following is probably to harsh self-criticism and too much into econometric stuff}
This definition of the control group is obviously problematic. Question: deal with this here or have a paragraph in the end that deals with shortcomings of the identification section? No matter where, it should go something like this: 
This definition of a control group is potentially problematic as it also includes households that already received their rebate in a prior period. This might bias our results if there are long term effects of receiving the rebate that spill into the next period. Also, it leaves the consumption change from t-1 to t in the control group biased in case households receiving the rebate in t-1  actually respond positively to the rebate. Then the change in the control group is likely to be negative as in the next period households resume their smoother consumption pattern and therefore show a negative consumption change although this is due to the control group. This biases the effect of receiving the rebate upwards as even when reactions are small the divergence is larger than the reaction itself (thisreads shaky and I am not sure whether I should include such harsh criticism of my own work here).
However, Parker et al. (2013) (or was it MS?) include a one period lag of rebate in their analysis and find no significant role of it in determining consumption change. 
\\
In the following, we will slightly depart from Parker et al.'s identification strategy given their findings as well as our inclusion of more control variables. Namely, they argue that using the actual amount of tax rebate received can lead to an omitted variable bias. This concern arises because of how the stimulus is determined. First, it depends on the number of children as each dependent child adds 300 USD to the stimulus amount. Second, the stimulus excluding the child bonuses equals the household's net income tax liability (NTL; in the following also referred to as the net tax liability) as long as it is within the exogenously defined boundaries we discussed in Section 3.1. Parker et al. now argue that the NTL might also drive changes in consumption, rendering our treatment endogenous. Their solution is to instrument the amount received with a dummy variable that only signals whether the stimulus was received or not in the given quarter. While their results and the authors themselves suggest that this is not much of a concern, we decisively disagree with tier identification approach. The number of children is reported in the CEX and is easily controlled for as it is collected in each interview conducted. The role of the NTL is more complicated. Parker et al. do not control for any variable related to households income or salary, which in itself might already bias the results, but renders their estimation inconsistent as these two variables are very likely to impact changes in consumption and are obviously directly related to the net income tax liability. However, other than through the channel of income, we deem it highly unlikely that the net tax liability itself is driving changes in consumption. Especially, even if in other years it plays a role and can be seen as an anticipated income shock which size households do not know, in 2008 the net tax liability was encapsuled in the tax rebate, i.e. other than through the channel via tax rebate or via income it has no impact on the outcome by itself. Therefore, we argue for using the actual rebate amount since it has two advantages: for one, we have an additional source of variation and second it allows us to estimate the continuous treatment effect and interpret it as the actual MPC in dollar amount. \\
However, one drawback of the already mentioned lack of detailed documentation of household characteristics in the CEX is the fact that once we include financial variables - most importantly liquidity and salary - our sample size shrinks substantially because they are not consistently documented for each interviewed households. Although the DML framework achieves fast convergence rates even in cases in which the first stage predictions do not converge as rapidly, we have to keep this drawback in mind. However, contrary to Misra and Surico, we actually recover the conditional distribution of the MPC - obviously under the assumption that we control for any relevant confounders **(NOT A SENTENCE YET AND THINK ABOUT THIS IN MORE DETAIL)** - ****and contrary to Parker et al. we do not have to rely on defining our own cutoffs to detect heterogeneities. \\
\textbf{This can be an additional point combined with the fact that we are looking at change in 2008 - however, it is contrary to what we are showing, i.e. that households don't really account for future income shocks as they still react significantly (at least some)}
Most households will know from past experience in what range their net tax liability will fall and account for it in their decision making. 