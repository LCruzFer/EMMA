\section{Estimation and Results} \label{sec:estim_res}
We implement the partially linear model as presented in Section X.X to estimate the effect of receiving a tax rebate of $R_{it}$ on change of consumption $\Delta C_{it}$
\begin{align}
    \Delta C_{it+1}&=\theta(X_{it})R_{it+1}+g(X_{it}, W_{it})+\epsilon_{it} \label{eq:plm_C1}\\
    R_{it}&=h(X_{it}, W_{it})+u_{it} \label{eq:plm_C2}
\end{align}
Which variables are included as confounders $X_{it}$ and $W_{it}$ and which make up $X_{it}$ depends on our specification and is named in the section discussing the respective results. In each specification we include monthly dummies to account for seasonality. Moreover, these dummies capture any unobserved effects that might appear when households learn about the rebate, i.e. in line with Parker et al./Misra and Surico our estimation uncovers the effect of actually receiving the rebate and not the global rebate effect.

\subsection{Identifying the Income Shock}
When it comes to the identification of the income shock and its effect $\theta(X)$, we rather closely follow Parker et al. (2013) as do Misra and Surico (2014). We can exploit the design of the stimulus rollout to identify the income shock. The tax stimulus was paid out to households over several weeks as administrative and technological restrictions made it quite impossible to pay out all rebates at once. Instead the rebate receipt depended on the last two digits of a tax filers social security number. Therefore, we observe tax receipts at different points in time, which allows us to use all other households that received their rebate in a different quarter as the control group. This is the baseline reasoning in Parker et al. for their identification, however, we also depart from their approach by including several more control variables contrary to their baseline estimates. Parker et al. as well as Misra and Surico only include the households age and the change in family size - and squared terms respectively (only MS) - as potential confounders. While they also claim to include other confounders and find no different results, their results section is unclear on which specifications are shown and they argue that there is not much of a difference anyway **(at least I guess I remember it being this way). S**uch claims are problematic for several reasons among which are: they do not report any tests on whether these coefficients are indeed the same and excluded confounders can work in opposite directions in biasing the point estimates of interest letting it seem like there is no bias (**this needs to be a real sentence)**. However, one drawback of the already mentioned lack of detailed documentation of household characteristics in the CEX is the fact that once we include financial variables - most importantly liquidity and salary - our sample size shrinks substantially. Although the DML framework achieves fast convergence rates even in cases in which the first stage predictions do not converge as rapidly, we have to keep this drawback in mind. However, contrary to Misra and Surico, we actually recover the conditional distribution of the MPC - obviously under the assumption that we control for any relevant confounders **(NOT A SENTENCE YET AND THINK ABOUT THIS IN MORE DETAIL)** - ****and contrary to Parker et al. we do not have to rely on defining our own cutoffs to detect heterogeneities. 

Moreover, the inclusion of more control variables, especially information on households' salary and total income, allows us to identify the income shock using the actual rebate amount yielding two sources of variation across households: the timing of rebate receipt and the amount. **IS THE LATTER EVEN RELEVANT WHEN WE ESTIMATE A CONSTANT CATE??**
Parker et al. - and in the same line Misra and Surico - argue that the rebate amount itself could be endogenous. This is due to the fact the the rebate amount reflects household's net tax liability, which might affect the change in consumption across periods on its own but cannot be controlled for. However, we are convinced that this is not the case. We measure consumption on a quarterly basis and the earliest reporting dates of rebates are in May 2008. Therefore, consumption change for these households is measured with respect to their consumption from December 2007 to February 2008. Depending on when households file for income tax and receive their statement (**HOW DOES THIS WORK ACTUALLY?**) it is unlikely that their net tax liability affects change in consumption directly. Also, a positive net tax liability can be seen as a negative income shock and there is ample evidence that response to such - especially smaller - shocks is muted. Even more so when they are expected. Most households will know from past experience in what range their net tax liability will fall and account for it in their decision making. Therefore, an instrumental variable approach is not necessary and only leads us to lose valuable variation in the amount of rebate receipt. Moreover, these approaches do not uncover how much of each dollar is spent but rather what the amount of consumption change is for households from receiving the rebate in general.
