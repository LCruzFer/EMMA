\section{Estimation and Results} \label{sec:estim_res}
We investigate the heterogeneity of the Marginal Propensity to Consume by estimating the following partially linear model
\begin{align}
    \Delta C_{it+1}&=\theta(X_{it})R_{it+1}+g(X_{it}, W_{it})+\epsilon_{it} \label{eq:plm_C1}\\
    R_{it}&=h(X_{it}, W_{it})+u_{it} \label{eq:plm_C2}
\end{align}
where our outcome of interest is the change in consumption between two quarters, $\Delta C_{it}$ and our treatment is the rebate amount household $i$ receives. The choice of confounders $X_{it}$ and $W_{it}$ depends on the specification we estimate as does which variables we consider to be part of $X_{it}$ and thus have an effect on the treatment effect. Which variables are included in each specification is listed in Table X.X. We follow Parker et al. by including monthly dummies to account for seasonality and to capture any unobserved effects that might appear in periods in which households learn about the upcoming rebate. By cancelling these effects stemming from the anticipation of the treatment, our estimate represents the effect of actually receiving the rebate. \\ 
In total, we distinct between three different levels in our estimations: we investigate different outcomes $\Delta C_{it}$ by using the rich information on expenditure categories included in the CEX. With the term 'specifications' we distinct between the different sets of confounders X and W we use. Lastly, we estimate each outcome-specification pair twice: once using the linear and once using the causal forest second stage. Since our estimation procedure predicts MPCs and we retrieve standard errors, we can run hypothesis tests on whether the estimated response to the tax rebate is statistically significant for each household. \\
However, one drawback in our specifications including liquidity, salary and income is the already mentioned lack of detailed documentation of household characteristics in the CEX. Our sample size shrinks because they are not consistently documented for each interviewed households. This sample reduction can induce a sample selection bias because it is possible that households that answer questions on their liquidity are systematically different from households that provide informations on these measures. Although the DML framework achieves fast convergence rates even in cases in which the first stage predictions do not converge as rapidly, we have to keep this drawback in mind. 

\subsection{Identifying the Income Shock} \label{subsec:identification}
Since we use the same data and event to estimate the MPC our identification is based on the approach by Parker et al. (2013). The main factor is the design of the stimulus rollout, which we can exploit to identify the income shock. The tax stimulus was paid out to households over several weeks as administrative and technological restrictions made it impossible to pay out all rebates at once. Instead the date of rebate receipt depended on the last two digits of tax filers' social security number. These digits are randomly distributed and therefore the timing of the treatment is random rendering it exogenous from any household characteristics. Therefore, we observe rebate receipts at different points in time, which allows us to use all other households that received their rebate in a different quarter as the control group. \\
In the following, we depart from Parker et al.'s identification strategy given their findings as well as our inclusion of more control variables. They argue that using the actual amount of tax rebate received can lead to an omitted variable bias. This concern arises because of how households' stimulus payments are determined. Remember that the tax rebate directly depended on the number of children, which certainly affects the absolute level of households' expenditures, as each dependent child add 300 USD to the stimulus received. However, this is not a problem because we - as Parker et al - control for the number of children in each specification. The stimulus excluding the child bonuses equals the household's net income tax liability (NTL; in the following also referred to as the net tax liability) as long as it is within the exogenously defined boundaries we discussed in Section 3.1. Parker et al. argue that the NTL might also drive changes in consumption, rendering the treatment endogenous. Their solution is to instrument the amount received with a dummy variable that only signals whether the stimulus was received or not in the given quarter. While their results and the authors themselves suggest that this is not much of a concern \textbf{rephrase this sentence}, we decisively disagree with their identification approach. Parker et al. do not control for any variable related to households income or salary. These variables are without a doubt directly connected to our treatment because the NTL - i.e. how much a household owes in income taxes - is a function of the households income. Exlcuding these variables leads to an omitted variable biases causing inconsistent estimates. However, other than through the channel of income, we deem it highly unlikely that the net tax liability itself is driving changes in consumption. It might be possible that in other years the NTL plays a role for households income as it can be perceived as an anticipated income - or liquidity - shock.\footnote{Households usually should know that they will have to pay this/receive this because of past experience and because the NTL is also depending on how much income tax was already paid during the previous year.}However, in 2008 the NTL affected households via their tax rebate, i.e. it does not affect the consumption change through other channels than what is captured by the tax rebate. Therefore, we argue in favor of using the actual rebate amount since it has two advantages: for one, we have an additional source of variation and second it allows us to estimate the continuous treatment effect and interpret it as the actual MPC. 

\subsection{Main Results}
\textbf{Have to add details on ranges of estimates and how the distributions change in more detail}
We analyse our results in several steps and begin by looking at the empirical distribution of the estimated MPCs. Figure X.X shows the distribution of MPCs for the four main expenditure categories considered by Parker et al.: Food (FD), Strictly Non-Durables (SND) as defined by Lusardi (1996), Non-Durables (ND) and Total (TOT) expenditure. These categories are increasing in their level of aggregation, e.g. SND includes expenditures on food. A detailed list of all sub-components of each of these categories is listed in Appendix X.X. Here we only want to point out that the difference between SND and ND consumption categories are so-called 'semi-durables', such as health expenditures, which are not included in the SND category. \\
A single plot of the empirical distribution in Figure X.X is retrieved as follows. We slice the range between the minimum and maximum of the point estimates into 20 equidistant bins and calculate the share of estimated MPCs that fall into each bin. The x-axis signals the borders of the different bins and the y-axis shows the respective frequency. The blue bar signals what the total frequency of this bin is. To illustrate how many of these estimates are actually rejecting the null of a zero MPC, we calculate the share of point estimates that reject the null at the 10\% level within each bin. This is depicted by the red overlay over the frequency bars. I.e. a completely red bar implies that all observations within this bin are statistically significant whereas a bar that is only red up to half of its height signals that only half of the point estimates within this bar are statistically significant. The vertical dashed line marks the average CATE - the average treatment effect across all households - as a benchmark. The plot description notes whether this ATE is significant or not. \\ 
First, we have a look at global trends across all specifications and expenditure categories before we start taking a closer look at each category and estimation procedure. \\
We find strong support for heterogeneity in the Marginal Propensity to Consume. Plots in Figure X.X show a large variation in households' MPC. This underlines the importance of accounting for heterogenous responses to income shocks. The heterogeneity is similar to what Kaplan and Violante's theoretical model suggests and Misra and Surico's empirical findings. Namely, our results show a large mass of households having a Marginal Propensity to Consume that is closely distributed around 0 and for many households we cannot reject the null hypothesis of a zero MPC. On the other hand, there is a smaller share of households that show strong, significant responses. Contrary to Misra and Surico, these shares are smaller and the size of the significant MPCs is also higher. Table X.X depicts the shares of significant MPCs we estimate for each specification and model when we look at changes in non-durable consumption. As illustrated in Figure X.X., Table X.X shows that once we control for liquidity in Specification 3, the number of significant MPCs decline. This is independent of which estimator we use for the second stage as well as from the outcome we look at. \\
Most importantly though, we find that the ATE is always very close to zero but the individual point estimates show a completely different pattern. We see that the ATE falls into bins that have the highest frequency - which makes sense by construction - but across all estimations the respective bin never contains more than 15\% of all point estimates. This highlights the weak representativeness of the ATE and its inability to reliably assess the success of programs such as the 2008 tax stimulus. \\
Also we see that introducing more controls to our estimation reduces the spread of the point estimates no matter at which outcome and estimator we look at. The change is the most pronounced once we add liquid assets, income and salary as confounders to the estimation. \\
Curiously at first, the responses we find for total expenditures are unreasonably large for a bulk of individuals. This is probably due to the underlying composition of the total expenditure variable and our estimation approach. It is quite likely that some outliers within a specific spending category part of TOT are driving the learning behavior of our estimators \textbf{here mention outliers in vehicle purchases}. This is underlined by the fact that the causal forest estimator finds way larger responses than the linear based estimator. Nonparametric machine learning estimators are often performing weakly when they encounter unusual combinations of confounders and outcomes.  Also, it is important to note that in both estimators the spread of the CATE is drastically reduced once we control for liquidity, salary and income - variables we expect to be closely related to the MPC. Turning to the significance of the response, we see that adding more controls also reduces the amount of significant MPCs found; suggesting that prior specifications pick up signals of the confounding factors not included and interpret them as signals of the rebate. Concluding, it seems that our estimation procedure is quite sensitive to extreme outliers in the underlying consumption categories as for total expenditure we find extreme responses as laid out above. \\
This notion becomes more clearly when we turn to the non-durable and strictly non-durable goods. Excluding large durable categories such as *new vehicles* immediately reduces our estimated MPCs and a range between ... and ... with most significant MPCs ranging around .... Moreover, looking at the ND category we see that in specification 3, which includes liquidity, the linear model fails to reject the null for all households. Interestingly though, the causal forest model still finds a small fraction of large significant MPCs. This difference suggests that there are non-linearities in the dependence of the MPC on the variables such as liquidity - and potentially with respect to their interactions - that are ignored by the linear model but detected by the causal forest. Accounting for these non-linearities reduces the noise in the point estimates and reveals significant MPCs where the linear analysis fails to pick up any significant MPCs. Comparing this to the Strictly Non-Durable category there is little difference in the distribution and significance of the recovered MPCs. \\
Turning to the comparison of the two estimators, we see that the spread of estimated MPCs is substantially larger for estimates returned by the causal forest compared to the linear model. This hints to the fact that the linear model not including any interactions and non-linearities in the CATE, reduces the precision of the estimates. However, we have to keep in mind that the causal forest can result in more spurious estimates when handling outliers as extrapolating from unseen combinations will lead to imprecise predictions by the causal forest. \\
Throughout almost all estimations, we find a substantial share of households that show a negative MPC - a concept that is bounded by zero as its lower bound. This relates back to their argument that when using the tax stimulus we estimate a 'rebate coefficient' and not necessarily the MPC. However, the rebate coefficient can very well be negative as they show in estimations using their calibrated two-asset model. In their model, they explain the heterogeneous response to the 2001 tax stimulus by the government by distinciting between hosueholds that are welathy but only hold illiquid assets (wealthy hand-to-mouth) and households that have no liquidity and hold no illiquid assets (poor hand-to-mouth). \textbf{This is a sentence for the literature review; instead here only relate to this} In their two-asset model the households holding illiquid assets have to pay transaction costs to increase their holdings of the illiquid asset. Kaplan and Violante show that when these transaction costs are relatively low compared to the size of the income shock, households will choose to pay the costs and make a deposit once they receive the payment resulting in a negative effect on consumption. 