\newpage
\section{Data} \label{sec:data}
We use data collected by the Consumer Expenditure Survey (CEX) that is administered by the Bureau of Labor Statistics. Its main purpose is to provide information on the consumption preferences of US households to adjust the goods basket that is used to calculate various inflation measures (BLS, 2021). However, in an effort to understand the effects of the 2008 tax stimulus, \cite{parker_etal_13} added questions about these payments to the questionnaire between June 2008 and March 2009.  Due to its original purpose, the CEX provides a finely grained set of information on the type of goods households consume. This enables us to analyse on what kind of goods households with a non-zero MPC spend their rebate on. In the following, we briefly outline the stimulus program and describe the CEX data. 

\subsection{The 2008 Tax Stimulus Program} 
Due to the global financial crisis and the subsequent recession, the United States government passed the Economic Stimulus Act (ESA) in February 2008. With projected costs of more than 150 billion USD it was the largest relief program passed in the history of the USA up to this point. Next to the stimulus payments, which made up roughly two thirds of the program, the ESA also enacted other steps meant to provide economic relief such as enabling government owned entities (Fannie Mae and Freddie Mac) to buy up more mortgages. However, we only focus on the effects of the stimulus payments. \\
The rebate was paid out to any household that filed for income taxes. Households that fell beneath the minimum amount of income required to have to file for federal income taxes had to file for taxes anyway and were eligible for the minimum amount of rebate as long as they had a minimum annual income of 3,000 USD \textbf{lots of 'minimum' here}. Eligible households received their net tax liability as their rebate, however, the payment were bounded by a minimum of 300 and a maximum of 600 USD. For couples filing jointly the limits were 600 and 1,200 USD, respectively. Parents of children under the age of 17 received additional 300 USD per child. Additionally, the rebate was capped for high income households. The rebate was reduced by 5\% of the amount that the reported income exceeded 75,000 USD (150,000 USD for couples), which led the program to target mostly low to medium income households.

\subsection{Consumer Expenditure Survey} 
The CEX is a representative survey of households in the USA interviewing households about their consumption patterns on a quarterly basis. Once a household is selected to participate, they are interviewed a total of five times. The first interview is a baseline interview during which some general household characteristics, employment related variables and their stock of nondurable goods are documented.\footnote{The baseline interview has only been conducted until 2015. Since then the first interview covers these questions.} The next four interviews are administered every three months and households are asked to document their expenditures over the period since the last interview. The final interview collects data on global financial variables such as amounts saved in savings or checkings accounts, which we use as our measures for liquidity. After this interview, the household is rotated out of the CEX and replaced with a new one. Hence, each month of data documented in the CEX contains a different set of households as new ones are added and others are rotated out of the survey. Figure \ref{fig:cex_rotation} is taken from the CEX website and illustrates this procedure. Note that a household is defined as a Consumer Unit (CU), which can represent either a number of blood or legally related persons (e.g. foster children), a single individual - even if living with other people as long as the individual is financially independent - or unrelated people who are pooling their income. All information about a Consumer Units members are collected regarding their relationship to the reference person. This person is defined as the one named when asked who rents or owns the home. For personal traits such as age we follow the convention by \cite{parker_etal_13} and take the average of the characteristic of all CU members.
%%%%%
%FIGURE CEX ROTATION
\begin{figure}[t]
    \caption{CEX quarterly rotation procedure}
    \centering
    \includegraphics[width=.9\linewidth]{figures/CEX_rotation_table.png}
    \fnote{Columns show number of interview and a letter signals a specific household. Source: \url{https://www.bls.gov/opub/hom/cex/data.htm}}
    \label{fig:cex_rotation}
\end{figure}
%%%%%
\\ It is important to highlight the limitations set by the usage of CEX data. As mentioned, the main objective of the CEX is to assess what goods the average household consumes to create the goods basket for inflation measurements. This focus results in a lack of interest in a dense documentation of household characteristics and income related variables. For example, the lack of asking for liquidity related measures in each quarter prevents us from controlling for changes in liquidity but we can only control for households overall self-reported levels of liquidity. Also, the variables collected are only crude measures for liquidity. \\ 
\textbf{if doing liquidity check then refer to it in paragraph above in the end}
\\
While this is a disadvantage in comparison with other data sources, the CEX's richness in information on consumption behavior is unmatched. Keeping in mind the risk of measurement error through the self-reported consumption measurement, the CEX enables us to analyse not only the MPC for overall consumption but to dissect it and see which goods drive responses and heterogeneity seen in higher level estimates.