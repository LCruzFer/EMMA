\section*{Literature}
They estimate a simple fixed-effects regression in which they interact their income schock variable with pre-defined dummies. Those dummies are based on continuous variables and created by choosing discrete cut-off points. However, this prohibits the detection of heterogeneous patterns that are not captured by the variables considered or are not inside the defined thresholds. Using the Parker et al. data, Misra and Surico (2014) replicate their approach but use quantile regression to analyse the heterogeneity in the MPC distribution. While quantile regression can be of service to detect heterogeneity in coefficients, it does not allow for the correct interpretation. The treatment effects they uncover are the effect of the income shock on the difference in consumption before and after for a respective quantile. However, this quantile does not need to include the same individuals. Hence, the quantile regression only uncovers shifts in the overall distribution but is silent on how specific individuals changed their consumption pattern - and hence the actual MPC. \\
Lastly, as Kaplan and Violante \textbf{(or who exaclty was it?)} point out, empirical analysis that use stimulus payments as a temporary income shock to identify the MPC might actually estimate another coefficient, which they coin the coefficient of rebate. They argue that the conditions of a stimulus payment as well as the overall economic conditions that lead to such a payment are too specific (\textbf{rephrase}) to 