\section{Understanding the roots of heterogeneity} \label{sec:roots_of_heterogeneity}
In the previous section we discussed the conditional average treatment effect of each individual given their specific set of characteristics. Similarly to prior contributions, we also looked at correlations between the significance of the estimated MPC and households characteristics to get a glimpse into which factors play a role in the MPC. However, this approach does not reliably tell us which variables really drive the response. The correlation might very well be spurious or driven by other factors that are correlated with the characteristic we are looking at. Therefore, it is more fruitful to look at measures that can help us identify what role a variable plays in our predicted MPCs. In case of specifications using the linear DML estimator, we know that this relationship is linear by construction since the CATE is defined as a linear combination of the single effects of interactions between treatment and the respective variable (see equation (\ref{eq:CATE})). However, the causal forest based approach will help us reveal whether there are any non-linear patterns underlying in the effect of characteristics on the MPC without assuming any functional form of these patterns. \\
For this, we turn to the Machine Learning literature, which has developed a number of tools to analyse the relationship between prediction and feature. Feature is a different term for control variables. In our setting these are the variables we condition on to find the CATE, i.e. variables contained in $X$. Since variables in $W$ are assumed to not impact the CATE they are not contained in the second stage and therefore play no role in predicting individuals' MPC. Machine Learning estimators such as random forests are blackboxes as they only provide predictions but stay quiet on which variables are important to arrive at this prediction. The literature has proposed multiple approaches that help quantify the role of a single feature, some of which we look at in the following. 

\subsection{Marginal and Partial Dependence Plots}
Two popular approaches are marginal plots (M-Plots) and Partial Dependence Plots (PDPs; Friedman, 2001). Both use the same general idea to quantify the impact of some feature $x_S$ on our predictions: we replace the value of $x_S$ of each observation with some value $v_1$. Then we fit our trained prediction model to this "counterfactual" dataset and take the average over all these predictions. For example, we predict for each individual what their MPC looks like if they had a certain age and average the predicted MPC. Then we continue with $x_S=v_2$ and so on, where the values $v_j$ are chosen from a grid along the distribution of $x_S$. The difference between M-Plots and PDPs is the distribution of all other features $X_C=X\setminus x_S$ we average over. In case of Marginal Plots, contrary to what the name might suggest, we use the conditional distribution of $X_C$ given $x_S$, $p_{X_C|x_S}$, to obtain the impact of $x_S$ on our prediction
\begin{align}
\hat{f}_M(x_S=v_j)=\int p_{X_C|x_S=v_j} m(x_S=v_j, X_C)dX_C, 
\end{align}
where $m$ is our predictor and $\hat{m}_M(v_j)$ is the effect at $x_S=v_j$. On the other hand, PDPs use the marginal distribution of $X_C$, $p_{X_C}$, 
\begin{align}
    \hat{f}_{PDP}(x_S)=\int p_{X_C} f(x_S, X_C)dX_C \label{eq:pdp}
\end{align}
where $\hat{m}_{PDP}(v_j)$ is the Partial Dependence of our predictor on $x_S$ at $v_j$. Using the marginal distribution of $X_C$ effectively "marginalizes out" the effect of any other variables than $x_S$ at some point $v$ and therefore reveals what impact $x_S$ has on our prediction at this point. Partial Dependence Plots are more common in the Machine Learning literature as M-Plots suffer from a severe weakness when features in $X_C$ are correlated with $x_S$. However, PDPs also fail to reliably uncover the effect of $x_S$ in such a setting. \\
To illustrate the issues arising in M-Plots and PDPs when features are correlated, let us consider a simply example. Lets say we have some predictive model $m$ that only depends on two predictors $x_1$ and $x_2$, which are positively correlated. To now calculate the M-Plot of $x_1$ at $v_1$ we use the conditional distribution $p_{x_2|x_1}$. In practice, we plug in $x_1=v_1$ for each observation that is within a specified neighborhood of $x_1=v_1$ (e.g. observations in the same quantile). Then we predict and average to obtain the M-Plot value at $x_1=v_1$. Repeating this procedure for other values $v_j$ then results in the M-Plot of $x_1$. However, because the two variables are correlated, we do not know which variable drives the observed effect - if $x_1$  is increased, the values of $x_2$  we use for our predictions also increase because we only use $x_2$ of observations that are close to having $x_1=v_j$. This problem is known as 'conflation'. \\
On the other hand, Partial Dependence Plots do not suffer from this problem because they use the marginal distribution of $x_2$. We use all observations of $x_2$ instead of only looking at a neighborhood in which $x_1=v_1$ and, therefore, do average out the effect of $x_2$ on our predictions. Since we use the same set of $x_2$ values at each point $v_j$, we know that changes in our predictions must stem from $x_1.$ Still, the PDPs are not a good tool when features are correlated and this is connected to the machine learning estimators we apply them to. These are nonparametric estimators that are usually very weak in predicting outcomes based on observations they have never seen before. This extrapolation however becomes necessary when we create the "counterfactual" dataset by setting $x_1=v_j$. By doing so, we effectviely create observations that are extremely unlikely or even impossible to observed in the real world because of the correlation the features. For example, in our data age and salary are strongly correlated, which is quite intuitive because once retired, households do not receive a salary anymore. When creating PDPs we ignore this fact and create households that have a high salary and are very old. The weakness in extrapolation leads the model to create weak predictions, which then severly bias the Partial Dependence Plots. (Apley and Zhu, 2020)  \\
Therefore, while PDPs do not suffer from theoretical drawbacks like M-Plots, in practice they are unable to uncover the effects of $x_1$ on our predictions in a stable manner because of the underlying predictive estimator. If the true model is indeed linear and we use a linear prediction method with the correct specifications of any interaction terms etc., then this extrapolation issue is unlikely to occur. Moreover, by construction, a linear predictor will results in linear Partial Dependence Plots. \textbf{Remember that in our linear DML approach, we assume that the CATE we estimate is linear in features $X$ and the second stage - the fitted model we actually investigate here - is simply a linear regression, which results in a linear PDP by construction as our predicted MPC is simply the sum of all coefficients for individual $i$ given their characteristics. $\rightarrow$ I am not so sure about this part yet} \\
Indeed, results of the partial dependence plots are rather spurious. They are reported in more detail in Appendix X.X, where we look at the PDPs for non-durable consumption. The effects have a high variation and point estimates out of line of the existing literature. \textbf{this is not a good reasoning of why I don't show them because they are too close to the ALEs - I guess}

\subsection{Accumulated Local Effects}
To circumvent issues arising in M-Plots and PDPs from correlated features, Apley and Zhu (XXX) propose Accumulated Local Effects (ALE). The extrapolation issue PDPs suffer from is bypassed by using the conditional distribution $p_{X_C|x_S}$ as we do in M-Plots. As with M-Plots we use the conditional distribution to bypass the extrapolation issues that PDPs suffer from. The 'conflation' effect that results from this is tackled by not using average predictions at $x_S=v_j$ but rather the average marginal change in predictions at this point. We apply this by using the partial derivative of our predictor $m$ with respect to $x_S$ at the point $v_j$. Although many machine learning methods such as tree based learners have no concept of a gradient, Apley and Zhu are able to derive proofs for non-differentiable functions $m$ (see Section X.X) and further does this not play a role when it comes to the ALE estimation. The ALE is then defined as
\begin{align}
\hat{f}_{S, ALE} (x_S)=\int_{z_{0, S}}^{x_S} E_{X_C|X_S=x_S}[\hat{f}^S(X_S, X_C)|X_S=z_S]dz_S - constant, \label{eq:ale}.
\end{align}
Looking at this equation step-by-step reveals how the ALE recovers the effect of $x_S$ on our predictions even when features are correlated. As already mentioned, the ALE avoids 'conflation' by using the partial derivative of $m$, where we have $m^S=\frac{\partial m}{\partial x_S}\rvert_{x_S=v_j}$ as the partial derivative of $m$ evaluated at the point we want to find the ALE for. Since we only look at an infinitesimally small change, this change in $x_S$ will not affect the features that are correlated with it in $X_C$ unless the correlation is extremely high. In our analysis we would want to avoid this case anyways to avoid problems in the estimation itself (e.g. multicollinearity). Once the changes in prediction are obtained for each observation, we average them over the conditional distribution, i.e. only using observations that are within a neighborhood of $x_S=v_j$ and actually exist. Now we have the average local effect, but we are interested in how $x_S$ affects our predictions and not how it affects changes in predictions. Thus, we simply integrate over all local effects up to $x_S=v_j$, where $z_{0, S}$ is the lower bound of the distribution of $x_S$. \\
To estimate the ALE we use the following estimator, which illustrates the procedure in more intuitive terms: 
\begin{align}
\hat{\tilde{f}}_{j, ALE}(x)=\sum_{k=1}^{k_j(x)} \frac{1}{n_j(k)}\sum_{i:x_{i,j}\in N_j(k)}[f(z_{k,j}, x_{i,\setminus j})-f(z_{k-1,j}, x_{i,\setminus j})] \label{eq:ale_estim}
\end{align}
The intuition behind the estimator is straightforward. First, we bin our data into $n_b$ bins based on quantiles of the distribution of $x_S$. To mimic the marginal change represented by the partial derivative $m^S$ in \ref{eq:ale} we make two predictions for each individual. For an observation $i$ that falls in bin $k$, we predict its outcome with $x_S=z_k$ and $x_S=z_{k-1}$, where $z_k$ represents the upper bound value of quantile $k$. We then averages over all individuals that fall within this bin $k$ and finally accumulate all predicted differences from the lowest bin up to bin $k$. Only looking at individuals within a neighborhood $N(k)$ - effectively observations in the same quantile - accounts for the conditional distribution used in \ref{eq:ale}. \\
As a last step, Apley and Zhu propose to center the effect around the average of all ALEs such that the mean effect is zero. Thus, the ALE has to be interpreted relative to the average prediction and it shows whether for a given $x_S=v$ the effect of $x_S$ is above or below the average prediction. I.e. whether $x_S$ affects our predictions at $x_S=v$ more than it does on average. In practice, the $constant$ in (\ref{eq:ale}) is replaced by $$average term here$$. \\
Note that we yet cannot say something meaningful about the statistical significance of these results. Most fields are only interested in the predictive power of machine learning methods and to understand how these predictions are achieved, but there is no notion of statistical significance in these settings. Therefore, a specific approach to quantify uncertainty of these measures has not yet been developed. A deeper look into this topic is, however, out of the scope of this work. To briefly dive into the topic of statistic significance, we use a bootstrapping based approach. We simulate the ALEs for $n_{bootstrap}$ samples. These create an empirical distribution (need at least e.g. 500-1000) on which basis we calculate pseudo-standard errors. Figure X.X reports the Confidence Intervals using the reverse percentile approach (see Appendix X.X) and the mean point estimates in each bin. We see that these bands are very wide in certain parts - especially in areas where there is a small number of observations. We strongly encourage a deeper investigation of the statistical properties of ALEs and a potential way of quantifying their uncertainty in a more rigorous way. While the ALEs show us the relationship between a specific variable and our predictions, understanding whether this relationship is statistically significantly different from playing no role would be a major improvement. \\
One weakness of ALE is further that they are a global measure, i.e. they do not help to uncover heterogeneity in the reaction of the prediction to a single variable. One method that can unbeil such heterogeneities are Individual Conditional Expectations, but they suffer from the same conceptual problems as Partial Dependence Plots. Next to investigating the role of uncertainty in the measures presented in this section we therefore also urge the development of theoretical foundations of such a measure. For now, we account for heterogeneity by plotting the unaveraged ALEs of each individual. To avoid overplotting, we only plot the households at the quintiles of the ALE distribution. (\textbf{delete last part of this if not doing it!!})

\subsection{Results}
Figure X.X plots the Accumulated Local Effects.