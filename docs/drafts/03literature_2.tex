\section{Literature Review} \label{sec:lit}
The literature investigating the size of the MPC and potential heterogeneity can be summarized in three different strands. The first one uses quasi-experimental settings to exploit variation in income to estimate households' MPC. The second uses surveys that explicitly question participants about their MPC - be it out of actual or hyopthetical income shocks. Lastly, a vast literature focuses on building sophisticated macroeconomic models that are calibrated to match real world data and subsequently estimate the MPC agents experience in these models. In this section we briefly summarize the findings in all three and additionally discuss two studies - Parker et al. (2013) and Misra and Surico (2014) - in more detail as they investigate MPC heterogeneity using the same data as we do. \\
Quasi-experimental settings appear all the time in the real world, e.g. in case of a specific policy being implemented or another exogenous shock happening. Researchers interested in MPC heterogeneity focus on shocks that alter the income of a household rather abruptly. For example, \cite{fagereng_etal} use panel data from Norwegian administrative data on lottery winners. As they claim, most Norwegians participate in this state lottery, while the chances of winning are so low that eventhough households actively participate, winning can be seen as an unanticipated income shock. They find that households winning the lottery spend almost half of their win within one year and 90\% after 5 years. They also find evidence that supports the liquidity channel and the life-cycle hypotheses described in Section \ref{sec:intro}. Once controlling for household's balance sheet and general characteristics, liquidity and age are the only variables correlating significantly with MPC. Moreover, they show that responses in consumption drop sharply in the amount won. However, their approach suffers from two issues: one, they look at a rather long-term horizon. While this might still be interpreted as households' MPC most policy making issues are related to more short-term behavior. Second, their data does not measure spending directly but they have to construct it out of the households balance sheet data. Hence, any cash spendings have to be inferred into a spending category of which they only have two - durables and non-durables. \cite{golosov_etal} also analyse lottery winners to more granularly estimate households' reaction. Using U.S. data they are able to identify not only spending behaviour but also substitution effect dynamics when it comes to labor earnings. Indeed they find that lottery winners reduce their labor earnings by 50ct, labor taxes paid go down by 10ct and households spend 60ct of each dollar they win. Age is again negatively correlated with the MPC. Furthermore, they investigate the heterogeneity in the latter along the income distribution by cutting their sample into quartiles based on the latter. They find that in lower quartiles the MPC is substantially higher than in the highest quartiles. While their procedure of sample cutting is common in the literature, it is an inconvenient procedure to detect heterogeneity as it only allows the authors to analyse differences between their pre-defined groups. Hence, in case heterogeneity is strongest between other groups, their findings underestimate or even completely miss patterns in the data. \\
Gelman et al. (2018) use the government shutdown in the U.S. as a transitory liquidity shock. Hence, contrary to other literature they only estimate how liquidity changes the consumption behavior and not the MPC directly. Still, their setup allows them to disentangle the pure effect a liquidity shock has on consumers spending as government workers receive a payback of their wage once the government shutdown is over. Hence, there are no changes in expected income. Meanwhile, studies using income shocks cannot quantify what effect stems from the liquidity channel and what stems from changes in expected income. Their findings highlight that low liquid households react more to a negative liquidity shock as they have no assets to fall back on. Low liquid government workers started postponing their credit card payments, while simultaneously increasing the amount spend using them. \textbf{probably only add very short inside of this as not so much related to raw MPC and little/bad heterogeneity investigation} \\
The second strand of literature uses survey data from field surveys that question households about potential or actually relaized income shocks and how their reaction looks like. \cite{bunn_etal} use (\textbf{use twice here}) data collected by the Bank of England to assess the asymmetry that we expect in households' reaction depending on the sign of the income shock. As mentioned before, the liquidity channel suggests that an unanticipated shock calls for a stronger reaction if its negative. Indeed, the authors are able to provide ample evidence for such a reaction with their estimated MPC out of a negative shock being being between 5 to 12 times as high as the reaction to a positive shock. The balance sheet data their source provides also enables them to show that borrowers show a more pronounced asymmetry and reaction, which is also in line with the theoretical mechanisms of the liquidity channel. This also holds for households that face some kind of liquidity constraint. Additionally, Bunn et al. are capable of replicating their estimated MPCs in a model with households at the borrowing constraint. These findings are further underlined by Christelis et al. (2019) who use Dutch data for a similar study. Summarizing these studies strongly suggest the existence of the mechanisms related to households at their borrowing constraint and precautionatry saving motives (\textbf{the latter must be elaborated on in intro}). 

\subsection{2008 Tax Stimulus}
Lastly, we want to elaborate in more detail two already mentioned quasi-experimental studies: \cite{parker_etal_13} and \cite{ms_14}. These two studies have been quite influential when it comes to studying MPC heterogeneity and use the same dataset as we do. Therefore, we want to lay out their approache sin detecting heterogeneity in greater detail here and highlight the major advantages our estimation approach offers. Compared to the general literature, we are the first to focus on the form of the heterogeneities reported in the literature so far (\textbf{check whether this isn't too bold of a claim}). \\
Thanks to \cite{parker_etal_13}, the 2008 and 2009 BLS added questions about the tax stimulus to the CEX survey. \cite{parker_etal_13} use this data to first estimate a simple homogenous model controlling for age and the change in the family size using Ordinary Least Squares (OLS). Meanwhile, they set off to detect heterogeneity using interaction terms. More precisely, similar to \cite{golosov_etal} they create dummies that signal to which quartile a household belongs, where the quartiles are defined based on the distribution of various variables. The interaction terms then capture systematic differences between households across these quartiles. However, as we have already pointed out, this procedure can be quite problematic. It increases the potential to miss substantial heterogeneity across households. For example, consider the case where the largest heterogeneity is between the top 10\% and the top 20\% of households along some distribution. The approach using sample splitting or dummy variables is not capable of detecting these heterogeneities as those households belong to the same category. Additionally to this problem, using an OLS estimator, \cite{parker_etal_13} completely ignore any panel dynamics that might take place. Meanwhile, individual level fixed effects are identified as a strong potential channel that drives heterogeneity in MPC. The results by \cite{parker_etal_13} are therfore potentially biased (\textbf{two times potential}). \\
To adress the heterogeneity issue, \cite{ms_14} follow a new approach: they employ a quantile regression estimator to estimate the heterogenous response of households to receiving the rebate. The quantile regression estimator is similar to the OLS but estimating conditional quantiles instead of the conditional mean. The authors therefore claim to estimate the conditional MPC out of the rebate for any quantile of consumption change. However, there are severe dissimilarities with what they are actually estimating and what they interpret. QR estimates how much a variable affects the outcome at a given quantile, e.g. the 10\% quantile, of the conditional distribution of the outcome. 

\subsection{Channel description}
\textbf{This will be pushed/worked into another section: decide whether at beginning of literature review or in introduction; for now here only to formulate something in a more consistent manner than what's in the intro right now} \\
The theoretical literature has identified several channels which drive MPC heterogeneity. The two most prominent ones are life-cycle dynamics and liquidity. The former is driven by a consumer's age and the associated fluctuation in income. As data consistently shows (sources), consumption follows a hump shape over the life-cycle. In the case of liquiditiy, its role is linked to the nature of the income shock and completeness of the credit market. If a positive income shock is anticipated, households that are already close to or at their borrowing constraint cannot borrow new funds to smooth consumption in anticipation of a higher future income. Thus, once the shock realizes, we will observe an increase in consumption - although if we follow the PIH, this increase is rather small as the additional income is spread out over all future periods. In case of a negative anticipated shock, saving is always possible for any household and hence we will not see a reaction once the shock realizes. E.g. \cite{bunn_etal} document this asymmetry depending on the sign of the shock (\textbf{they document this for unanticipated shocks, shift back}). Thus, more liquid households react less to a positive anticipated shock in comparison with liquiditiy constrained ones. In contrast, in case of an unanticipated shock, we expect the opposite. Think of an agent that is temporarily out of work and has no liquid wealth at their disposal. In case of a negative shock, the agent is forced to adjust their consumption behavior downward. Meanwhile, a positive shock will always be saved and stretched over future periods, no matter the level of households' liquidity. However, these theoretical predictions are made within a permanent income framework in which households try to smooth consumption over time.\\ 
\\
What channels are there that I want to highlight? 
\begin{enumerate}
    \item precautionary saving motives
    \item borrowing/liquidity constraint
    \item consumption-retirement puzzle/finite horizon/life-cycle dynamics
\end{enumerate}
Thus far, the literature has identified three main channels that drive MPC heterogeneity. The most prominent one, that has also been extensively documented in empirical work, is the liquidity channel. This channel is closely related to the interplay between incomplete/imperfect capital or credit markets and households that are close to or at their borrowing constraint. The theoretically predicted reaction of these households strongly depends on the kind and direction of the income shock. Shocks can either be anticipated or unanticipated with a positive or negative sign, respectively. The most straightforward case is an negative anticipated income shock. As soon as households learn about it, they will start saving part of their income to be able to smooth consumption despite this shock realizing. Since saving is always available to all households, we expect no significant reaction by any household in the period in which the shock actually takes place. However, in cases of positive shocks, the borrwing constraint - and hence the available liquidity - becomes a major driver. In the optimal setting all households will start increasing their spending slightly to smooth consumption as soon as they are informed about the upcoming shock. However, due to various factors, some households at their borrowing constraint, will not be able to borrow (enough) to smooth consumption over time. Once the shock realizes these agents will increase their consumption more than others as they are only now able to consume the exta amount of income. \\ 
For unanticipated shocks, the world looks different. Here, in case of a positive shock, households will want to save some of it to smooth consumption. As this is always possible, there is no heterogeneity in households' response. Meanwhile, in case of a negative shock, consider an agent that has no liquid wealth at their disposal. This agent is not capable of smoothing out the negative income shock and has to reduce their level of consumption immediately, while others can use their savings/increase their borrowing to dampen the contraction in consumption and spread it over several periods. \\ 

\textbf{Literature Notes}\\ 
Parker et al. (2013) 
\begin{itemize}
    \item three sources of variation exist: 
    \begin{enumerate}
        \item timingand type of payment
        \item amount 
        \item type of payment
    \end{enumerate}
    \item result: on avg. hosueholds spent 12-30\% of rebate on nondurable consumption goods $\rightarrow$ significant 
    \item life-cycle/PIH model (LCPIH) is rejected by these findings 
    \item LCPIH: no response to anticipated shocks at timing of arrival $\rightarrow$ borrowing/liquidity constraints may be main driver 
    \item prior research: larger payments may skew consumption/usage of rebate towards durables (Souleles (1999) finds significant increase in ND and D goods in response to larger payments (federal tax refund in springtime))
    \item problem data: assets are not measured in detail and frequently 
    \item find no significant response (interpret it anyway)
    \item keep in mind that tax rebate was disbursed during time of major economic downturn/turmoil
    \item look at relationship to owning house compared to renters $\rightarrow$ homeowners spend more than renters
    \item closest literature: Agarwal, Liu and Souleles (2007); Broda and Parker (2008); Bertrand and Morse (2009)
    \item Tax rebate 
    \begin{itemize}
        \item at least 300 (couples 600), at most 600 (1200) if 300+tax liability are above 600 
        \item at least 3000\$ qualifying income 
        \item phased out with income starting at 75000\$: 5\% reduction of amount gross income exceeds 75k (couples 150k)
        \item hh received a notice in advance of payment $\rightarrow$ anticipated shock!
    \end{itemize}
    \item CEX 
    \begin{itemize}
        \item questions were added from june 2008 to march 2009 
        \item use 2007 and 2008 waves of CEX (2008 data includes first quarter of 2009)
    \end{itemize}
    \item identification 
    \begin{itemize}
        \item use 
    \end{itemize}  
\end{itemize}