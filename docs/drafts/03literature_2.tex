\section{Literature Review} \label{sec:lit}
The literature investigating the size of the MPC and potential heterogeneity can be summarized in three different strands. The first one uses quasi-experimental settings to exploit variation in income to estimate households' MPC. The second uses surveys that explicitly question participants about their MPC - be it out of actual or hyopthetical income shocks. Lastly, a vast literature focuses on building sophisticated macroeconomic models that are calibrated to match real world data and subsequently estimate the MPC agents experience in these models. In this section we briefly summarize the findings in all three and additionally discuss two studies - Parker et al. (2013) and Misra and Surico (2014) - in more detail as they investigate MPC heterogeneity using the same data as we do. \\
Quasi-experimental settings appear all the time in the real world, e.g. in case of a specific policy being implemented or another exogenous shock happening. Researchers interested in MPC heterogeneity focus on shocks that alter the income of a household rather abruptly. For example, \cite{fagereng_etal} use panel data from Norwegian administrative data on lottery winners. As they claim, most Norwegians participate in this state lottery, while the chances of winning are so low that eventhough households actively participate, winning can be seen as an unanticipated income shock. They find that households winning the lottery spend almost half of their win within one year and 90\% after 5 years. They also find evidence that supports the liquidity channel and the life-cycle hypotheses described in Section \ref{sec:intro}. Once controlling for household's balance sheet and general characteristics, liquidity and age are the only variables correlating significantly with MPC. Moreover, they show that responses in consumption drop sharply in the amount won. However, their approach suffers from two issues: one, they look at a rather long-term horizon. While this might still be interpreted as households' MPC most policy making issues are related to more short-term behavior. Second, their data does not measure spending directly but they have to construct it out of the households balance sheet data. Hence, any cash spendings have to be inferred into a spending category of which they only have two - durables and non-durables. \cite{golosov_etal} also analyse lottery winners to more granularly estimate households' reaction. Using U.S. data they are able to identify not only spending behaviour but also substitution effect dynamics when it comes to labor earnings. Indeed they find that lottery winners reduce their labor earnings by 50ct, labor taxes paid go down by 10ct and households spend 60ct of each dollar they win. Age is again negatively correlated with the MPC. Furthermore, they investigate the heterogeneity in the latter along the income distribution by cutting their sample into quartiles based on the latter. They find that in lower quartiles the MPC is substantially higher than in the highest quartiles. While their procedure of sample cutting is common in the literature, it is an inconvenient procedure to detect heterogeneity as it only allows the authors to analyse differences between their pre-defined groups. Hence, in case heterogeneity is strongest between other groups, their findings underestimate or even completely miss patterns in the data. \\
Gelman et al. (2018) use the government shutdown in the U.S. as a transitory liquidity shock. Hence, contrary to other literature they only estimate how liquidity changes the consumption behavior and not the MPC directly. Still, their setup allows them to disentangle the pure effect a liquidity shock has on consumers spending as government workers receive a payback of their wage once the government shutdown is over. Hence, there are no changes in expected income. Meanwhile, studies using income shocks cannot quantify what effect stems from the liquidity channel and what stems from changes in expected income. Their findings highlight that low liquid households react more to a negative liquidity shock as they have no assets to fall back on. Low liquid government workers started postponing their credit card payments, while simultaneously increasing the amount spend using them. \textbf{probably only add very short inside of this as not so much related to raw MPC and little/bad heterogeneity investigation} \\
The second strand of literature uses survey data from field surveys that question households about potential or actuelly relaized income shocks and how their reaction looks like. \cite{bunn_etal} use such a setting to investigate the asymmetry in MPC out of positive and negative income shocks that we have seen with respect to the liquidity channel. 

\subsection{2008 Tax Stimulus}