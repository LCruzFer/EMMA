\section{Introduction} \label{sec:intro}
How do households respond to income shocks and how do their responses differ given their personal characteristics and economic circumstances? These questions are not only at the centre of a wide academic debate in economics but also of major importance for policy makers. While the former revolves around verifying or neglecting the main mechanisms of the Permanent Income Hypothesis (PIH), the latter are interested in improving government transfers to more efficiently use public funds. These two sides have sparked many investigations using a wide array of approaches to quantify hosueholds' responses to income shocks - the Marginal Propensity to Consume (MPC). At the centre of macroeconomics since Keynes introduced it at the heart of his General Economic Theory, it quantifies how much households will spend on consumption of each dollar they receive from an income shock. While research has long focused on testing whether the MPC out of income shocks is zero in general and thus in line with the PIH, the literature has shifted its focus over the last decade. Most studies support the notion of an average zero MPC, but more recent evidence suggests that for specific groups the response is significantly different. \\
Empirical research related to the MPC and its heterogeneity has used several settings to identify income shocks. One of the most prominent is to use natural experiments in which households receive exogenous income shocks. Following \cite{parker_etal_13} and \cite{ms_14}, we exploit the 2008 tax rebate in the USA to estimate households' MPC using data collected by the Consumer Expenditure Survey (CEX). Similar to these two prior studies, we are able to use the rich information on consumption the CEX provides to not only identify heterogeneities in the overall MPC but also to analyse which categories of consumption goods households spend their rebate on. However, our econometric approach sets us apart as it is more sophisticated and more precise in detecting heterogeneities compared to any contribution we are aware of. \\ 
We use the Double Machine Learning Framework (DML) developed by \cite{DML2017} to estimate individual level point estimates of the MPC as well as standard errors for each household. This enables us to run hypothesis tests on whether a household's MPC is statistically significantly different from zero. The DML allows us to estimate the conditional average treatment effect (CATE) of the tax rebate on changes in consumption. Meanwhile, thanks to the semi-parametric nature of the DML framework we can use reliable Machine Learning models to control for any confounding factors without having to define their relationship with the outcome. Moreover, one of the two estimators we employ retrieves the CATE without assuming a specified relationship between variables we condition on (X) and the CATE itself. I.e. we do not assume that the CATE is linear in variables we condition on but let this relationship unspecified. \\
Our results underline the heterogeneity of the Marginal Propensity to Consume out of the tax stimulus documented in Parker et al. and Misra and Surico. We find the a large mass of households shows no significant reaction upon receiving the stimulus payment, whereas a smaller fraction of households shows strong and significant reactions above an estimated MPC of 0.5. Our analysis suggests that liquidity is indeed the main driver of MPC heterogeneities and that low liquid households are the ones reacting most sharply. Contrary to prior work, our estimated CATE does not rely on specifiying subsets of the data across which we assume heterogeneity to exist. We employ modern methods to quantify the effects of single variables on the estimated MPCs to understand the role of these characteristics. Our non-linear estimators suggest the heterogeneities presented in other work are imprecise. Next to our contribution to the MPC literature by providing a empirically more robust analysis, we also see our contribution in introducing modern and flexible estimation approaches to the macroeconomic literature. Frameworks such as the DML offer a gateway to new methods and identifications in the macroeconomic literature. We stress the importance of further research into the theoretical and applied nature of these procedures and their usage in more settings. \\
It is important to highlight that the Economic Stimulus Act in 2008 was signed into law by President Bush in February 2008. The tax rebate payments, which were part of this policy, started in April of the same year and are therefore an anticipated income shock. This becomes relevant when investigating the role of various factors, especially liquidity. Also, the tax rebate was disbursed to US taxpayers during a time of national and global economic downturn. Many households receiving the stimulus might have been in economic turmoil when receiving the payment and actually spend it to cover regular expenses that they otherwise would not have been able to cover (e.g. rent, utilities or other necessities of daily life). However, \cite{parker_etal_13} emphasise that some rebates were reported to be received outside of the disbursal window, which suggests that the income shocks might not have been anticipated and only noticed after their arrival. \\
The fine-grained consumption data of the CEX allows us to identify what kind of goods households consumed and what they spent their stimulus money on. As Kaplan and Violante (2014) note, the tax stimulus is anticiapted and is subject to these special circumstances. Therefore, one might also speak of our estimated coefficients as a 'Propensity to consume the rebate' or 'rebate coeffcient', which is not necessarily equivalent to households' overall MPC. While a government stimulus program might not be perfectly appropriate to verify theoretical models concerned with the MPC, providing evidence on their effect on individuals is of major importance for future policy making. When economic relief is urgent, non-targeted stimuli can present a viable option, targeted transfers can play a major role in many policy settings. Thus, understanding what households spend government transfers on and what households actually use them for consumption, is an important part of efficient policy-making. Additionally, quantifying the effectiveness of untargeted transfers is necessary to assess whether they are actually helpful to boost the economy. Aggregate estimates of the MPC suggest that this is not the case, but taking a closer look and adjusting for household characteristics reveals heterogeneities and effectiveness of these transfers. \\ 
The rest of the paper is structured as follows: Section \ref{sec:lit} summarizes the theoretical and empirical literature on MPC heterogeneity. Section \ref{sec:data} discusses the data source and challenges connected with it. The empirical methodology we use is described in Section \ref{sec:methodology}, while Section \ref{sec:estim_res} presents the identification and estimation results of the MPC. We further investigate sources of heterogeneity in responses in Section \ref{sec:roots_of_heterogeneity}. Section \ref{sec:conc} concludes.