\section{Introduction} \label{sec:intro}
The Marginal Propensity to Consume (MPC) is at the centre of the macroeconomic model introduced by John Maynard Keynes in his General Economic Theory. Eversince its introduction, the role and size of the MPC has been subject to debate. While Keynes declared the MPC to be meaningfully different from zero, the permanent income hypothesis developed by Milton Friedman - a corner-stone of modern macroeconomics - declares it to be irrelevant to current consumption decisions and, thus, irrelevant to short-term economic policy making. However, both are wrong and right at the same time. Both are wrong and right at the same time. \\
The more recent literature has shifted the focus from assessing the reaction to income shocks across all households to a more nuanced view that takes into account heterogeneity across households. This follows the overall trend in macroeneomic theory focusing on heterogeneous agents that has taken place over the course of the last two decades. The theoretical as well as empirical literature is now working on improving the understanding what drives a households reaction to an income shock and whether there are substantial differences across these reactions. \\
Similar to previous quasi-experimental contributions to the empirical literature I use the 2008 U.S. tax stimulus program to quantify the MPC. The data is taken from the Consumer Expenditure Survey (CEX), which also collected information on timing and size of the rebate households received. This effort was made thanks to \textbf{add correct paper here (Parker et al 2013 or something)} who added those question to the regular questionnaire of the CEX. We use the same cleaned dataset as Parker et al. (2013) and Misra and Surico (2014) to compare our results to two of the most prominent papers investigating the heterogeneity of the MPC using the tax stimulus. However, compared to these two already existing works, our estimation procedure is less restrictive and far more rigorous. More precisely, we use a Double Machine Learning (DML) approach capable of detecting non-linear heterogeneities and controlling for confounders without any assumptions on their relationship with consumption and the rebate. Hence, our contribution is twofold: for one, we estimate the conditional MPC out of the tax stimulus in the most precise and rigorous manner thus far. Second, we use an estimator that exploits the power of machine learning methods for causal inference and contribute to the wider understanding and promoition of this method among applied researchers. Machine Learning predictors are powerful tools when it comes to handling large data and/or complex relationships between variables without any specification of those. \\
The theoretical literature has identified several channels which drive MPC heterogeneity. The two most prominent ones are life-cycle dynamics and liquidity. The former is driven by a consumer's age and the associated fluctuation in income. As data consistently shows (sources), consumption follows a hump shape over the life-cycle. In the case of liquiditiy, its role is linked to the nature of the income shock and completeness of the credit market. If a positive income shock is anticipated, households that are already close to or at their borrowing constraint cannot borrow new funds to smooth consumption in anticipation of a higher future income. Thus, once the shock realizes, we will observe an increase in consumption - although if we follow the PIH, this increase is rather small as the additional income is spread out over all future periods. In case of a negative anticipated shock, saving is always possible for any household and hence we will not see a reaction once the shock realizes. E.g. Bunn et al. (2018) document this asymmetry depending on the sign of the shock. Thus, more liquid households react less to a positive anticipated shock in comparison with liquiditiy constrained ones. In contrast, in case of an unanticipated shock, we expect the opposite. Think of an agent that is temporarily out of work and has no liquid wealth at their disposal. In case of a negative shock, the agent is forced to adjust their consumption behavior downward. Meanwhile, a positive shock will always be saved and stretched over future periods, no matter the level of households' liquidity. However, these theoretical predictions are made within a permanent income framework in which households try to smooth consumption over time. It is important to hihglight that in our setting, households experience an anticipated - at least for most - positive income shock. Therefore, following the previous arguments, theory would expect older households to react less, but liquidity not being a major driver. However, the tax rebate was disbursed to US citizens (\textbf{maybe citizens sounds like only those got the rebate, which is not true 100\%}) during a time of national and global economic downturn. Thus, many households receiving the stimulus might have been in economic turmoil when receiving the payment and actually spend it to cover regular expenses that they otherwise would not have been able to cover (e.g. rent or utilities). The fine-grained consumption data of the CEX allows us to identify what kind of goods households consumed and what they spent their stimulus money on. As Kaplan and Violante (2014) note, the tax stimulus is anticiapted and is subject to these special circumstances. Therefore, one might also speak of our estimated coefficients as a 'Propensity to consume the rebate' or 'rebate coeffcient', which is not necessarily equivalent to households' overall MPC. We compare our estimates with the range found in the literature using different income shock sources to get a grasp of whether this difference might play a role and for what households it does. \\ 
\textbf{section on results here}
We show that indeed both these channels play a role in the heterogeneity of households' response to the 2008 tax stimulus. Similar to the existing literature we find... However, additionally we are able to show that the heterogeneity is not only linear/indeed linear... \\ 
The rest of the paper is structured as follows: Section \ref{sec:lit} summarizes the theoretical and empirical literature on MPC heterogeneity. Section \ref{sec:data} discusses the data source and challenges connected with it. The empirical methodology we use is described in Section \ref{sec:methodology}, while Section \ref{sec:estim_res} presents the identification and estimation results. Section \ref{sec:conc} concludes.