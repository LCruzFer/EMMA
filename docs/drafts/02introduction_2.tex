\section{Introduction} \label{sec:intro}
% \textbf{Points to mention}
% \begin{itemize}
%     \item focus more on policy maker view and why it's crucial to understand responses
%     \item conditional Treatment Effect is of first-order importance for policy makers as in other settings more targeted transfers might be suitable and it's important to understand what people spend their transfer on (analysing the different good categories) $\rightarrow$ compare TE for differen sub-categories for same subset of households (e.g. bottom 20 in total reaction to see how their reaction splits up)
% \end{itemize}
How do households respond to income shocks and how do their responses differ given their personal characteristics and economic circumstances? These questions are not only at the centre of a wide academic debate in economics but also of major importance for policy makers. While the former revolves around verifying or neglecting the main mechanisms of the Permanent Income Hypothesis (PIH), the latter are interested in improving government transfers to more efficiently use public funds. These two sides have sparked many investigations using a wide array of approaches to quantify hosueholds' responses to income shocks. This Marginal Propensity to Consume (MPC) - at the centre of macroeconomics since Keynes introduced it at the heart of his General Economic Theory - quantifies how much households will spend on consumption of each dollar they receive from an income shock. While research has long focused on quantifying whether the MPC out of income shocks is zero and thus in line with the PIH, the literature has seen a shift of focus over the last decade. On average most studies support the notion of a zero MPC, however, more recent evidence suggests that for specific groups the response is significantly different. \\
Empirical research related to the MPC and its heterogeneity has used several settings to identify income shocks. One of the most prominent is to use natural experiments in which households receive exogenous income shocks. Following \cite{parker_etal_13} and \cite{ms_14}, we exploit the 2008 tax rebate in the USA to estimate households' MPC using data collected by the Consumer Expenditure Survey (CEX). Similar to these two prior studies, we are able to use the rich information on consumption the CEX provides to not only identify heterogeneities in the overall MPC but also to analyse which categories of consumption goods households spend their rebate on. However, our econometric approach sets us apart as it is more sophisticated and more precise in detecting heterogeneities compared to any contribution we are aware of.\\ 
Namely, the two main channels are life-cycle dynamics and liquidity. The former is driven by a consumer's age and the associated fluctuation in income. As data consistently shows (Attanasio and Weber, 2010), consumption expenditures follow a hump shape over the life-cycle rather than being roughly constant as we would expect under the PIH. This anomaly is often referred to as the retirement-consumption puzzle and connected to an increased amount of free time which allows households to reduce the cost of their consumption. Therefore, we would expect a reduction in the measured MPC in age. \\
In the case of liquidity, its role is linked to the nature of the income shock and borrowing constraints. If a positive income shock is anticipated, households that are already close to or at their borrowing constraint cannot borrow new funds to smooth consumption in anticipation of a higher future income. Thus, once the shock realises, we will observe an increase in consumption. On the other hand, saving is always possible for any household and hence we will not see a reaction once the shock realises in case of a negative anticipated shock. Thus, more liquid households react less to a positive anticipated shock in comparison with liquidity constrained ones. In contrast, in case of an unanticipated shock, we expect the opposite. Think of an agent that is temporarily out of work and has no liquid wealth at their disposal. In case of a negative shock, the agent is forced to adjust their consumption behaviour downward. Meanwhile, a positive shock will always be saved and stretched over future periods, no matter the level of households' liquidity. E.g. \cite{bunn_etal} document this asymmetry depending on the sign of the shock. \\
It is important to highlight that in our setting, households experience an anticipated positive income shock. The Economic Stimulus Act was signed into law by President Bush in February 2008 and payments of tax rebates started in April of the same year. Therefore, following the previous arguments, theory would expect older households to react less, but liquidity not being a major driver. However, the tax rebate was disbursed to US taxpayers during a time of national and global economic downturn. Thus, many households receiving the stimulus might have been in economic turmoil when receiving the payment and actually spend it to cover regular expenses that they otherwise would not have been able to cover (e.g. rent or utilities). However, \cite{parker_etal_13} emphasise that some rebates were reported to be received outside of the disbursal window, which suggests that the income shocks might not have been anticipated and only noticed after their arrival. \\
One major issue in the existing literature is the way how heterogeneity is measured. Most studies rely on either splitting their sample into smaller sub samples and estimating the MPC within each sample or use dummy variables that are defined by the authors based on some continuous variable. These approaches suffer from the severe issue that any heterogeneity that does not fall into this pre-defined pattern is not captured and will muddy the results of these investigations. In the worst case these procedures miss to pick up existing heterogeneity or missing patterns within these pre-defined subgroups. On the contrary, our Double Machine Learning (DML) approach allows us to estimate the conditional MPC out of the tax rebate of each individual household. Prior studies have to rely on looking at the correlation between their estimates and characteristics such as liquidity, but our setting enables us to calculate more sophisticated measures that capture the influence of each variable on the MPC. \\
The fine-grained consumption data of the CEX allows us to identify what kind of goods households consumed and what they spent their stimulus money on. As Kaplan and Violante (2014) note, the tax stimulus is anticiapted and is subject to these special circumstances. Therefore, one might also speak of our estimated coefficients as a 'Propensity to consume the rebate' or 'rebate coeffcient', which is not necessarily equivalent to households' overall MPC. We compare our estimates with the range found in the literature using different income shock sources to get a grasp of whether this difference might play a role and for what households it does. However, while a government stimulus program might not be perfectly appropriate to verify theoretical models concerned with the MPC, providing evidence on their effect on individuals is of major importance for future policy making. While in some cases when economic relief is urgent broadly defined, non-targeted stimuli might be a good option to pursue, targeted transfers can play a major role in many policy settings. \textbf{(rephrase)}\\ 

\textbf{this needs to be implemented somewhere}
By exact definition, the MPC is the reaction to an unanticipated, transitory income shock. However, in out setting, the shock cannot be fully seen as unanticipated as the tax stimulus payment was signed into law and therefore known to the public several months before the payments were conducted. However, in such cases to identify only the contemporaneous reaction of households the empirical literature (e.g. \cite{parker_etal_13}) .  \\ 

\textbf{section on results here}
We show that indeed both these channels play a role in the heterogeneity of households' response to the 2008 tax stimulus. Similar to the existing literature we find... However, additionally we are able to show that the heterogeneity is not only linear/indeed linear... \\ 
The rest of the paper is structured as follows: Section \ref{sec:lit} summarizes the theoretical and empirical literature on MPC heterogeneity. Section \ref{sec:data} discusses the data source and challenges connected with it. The empirical methodology we use is described in Section \ref{sec:methodology}, while Section \ref{sec:estim_res} presents the identification and estimation results. Section \ref{sec:conc} concludes.