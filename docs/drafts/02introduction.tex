\section{Introduction} \label{sec:intro}
\textbf{DON'T FORGET ABOUT LIFECYCLE MODEL!!!}
The Marginal Propensity to Consume (MPC) is at the centre of the macroeconomic model introduced by John Maynard Keynes in his General Economic Theory. Eversince its introduction, the role and size of the MPC has been subject to debate. While Keynes declared the MPC to be meaningfully different from zero, the permanent income hypothesis developed by Milton Friedman and a corner-stone of modern macroeconomics declares it to be irrelevant to current consumption decisions and, thus, irrelevant to economic policy making. However, both are wrong and right at the same time. More recently, the focus of research concerned with undertanding the MPC to guide policies - such as stimulus payments - has shifted to painting a more diverse picture of households' willingness to spend out of a transitory income shock. New, sophisticated models formalize the heterogeneity of agents in the macroeconomy, including their MPC. Additionally, empirical work has shifted from trying to prove an MPC of zero - or the opposite - to understanding the difference across households and allowing for heterogeneity in the MPC. \textbf{add two examples of channels - liquidity constraint and ...} \\
Using the 2008 tax stimulus as an exogenous income shock, my contribution to the empirical literature is twofold: First, I use a new and highly flexible estimation approach, that allows me to identify a wider range of heterogenous effects. The so-called Double Machine Learning Approach allows for a semi-parametric setup in which the functional form of any confounding factor does not have to be specified. Second, the literature using the data from the 2008 stimulus (or the general literature? \textbf{check!}) so far has investigated its effect (poor) methods that lack the advantage of the DML approach while at the same time not allowing to account for the panel data setting of the data. These approaches implicitly impose a strict exogeneity condition, while the Panel DML model is capable of accounting for possible effects of past characteristics on the change in current income. Therefore, I am able to identify the causal effect of the tax rebate on consumption change more clearly (or: actually identify it, but maybe to harsh). \\
I rely on data collected by the Consumer Expenditure Survey (CEX), which included a special part in the 2008 and 2009 surveys dedicated to the tax stimulus. This effort was promoted by Johnson, Parker and Souleles (2013; henceforth JPS) who quantify the effect of the tax stimulus on consumption changes. The data they use is publicly available and also used by Misra and Surico (2014; henceforth MS). Hence, to improve comparability with two of the more recent and prominent contributions, I use the data provided publicly by JPS as well. While both document some heterogeneity in the MPC, there are several drawbacks in their respective analysis. Meanwhile, the DML estimation allows me to identify household level point estimates and standard errors, allowing me quantify whether the estimated MPC is significantly different from zero for each individual to uncover which households actually experience a temporary increase in consumption due to a temporary income shock (\textbf{rephrase}). \\
\textbf{rewrite and put this somewhere else}
Understanding which underlying factors drive heterogeneity in the MPC is crucial for policy makers. While short-term untargeted tax-stimuli such as the one in 2008 are reasonable in times of economic crisis when time is short, targeted stimuli can improve the payoff of each dollar invested into an economic stimulus. \\
\textbf{add a brief summary/overview of what I find} \\
The rest of the paper is structured as follows: Section 2 summarizes the theoretical and empirical literature on MPC heterogeneity putting a focus on the issues concering JPS and MS analysis. Section 3 discusses the data source and challenges connected with it. The empirical methodoogy I use is described in Section 4, while Section 5 presents the results. Section 6 concludes.